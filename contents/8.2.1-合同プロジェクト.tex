\subsection{合同プロジェクト}
\par 後期は前期成果として出た仕様書をもとに開発に着手した.未来大は「Cool Japanimation」のAndroid,HTML5,サーバと「Rhyth/Walk」のAndroid,iOS,サーバ,ビジネスモデルの考案を担当した.神奈工は「Rhyth/Walk」のAndroidを担当した.長崎大は「Cool Japanimation」のHTML5,サーバを担当した.また,3大学合同でビジネスモデルの考案を行った.前期同様に毎週水曜のSkype会議を行って進捗を確認し,情報の共有を行っていた.各アプリケーション,プラットフォームごとでも話し合いを進めながら第二回合同合宿へ向けて開発を進めた.
\par 第二回合同合宿の初日は当初想定していた内容がうまく進まず,企業の方やOB・OGの方の支援を受けつつそれぞれのアプリケーションの魅力を伝えるためのデモシナリオを作成することになった.それと平行して現在のアプリケーションの開発進捗をまとめるとともに機能の取捨選択や優先度順位付けを行った.それらでまとまった話をスライドにまとめ,企業の方に発表した.質疑応答を行った末にデモシナリオと現状から最終発表までのToDoリストが完成した.
\par 本プロジェクトが学内で行う発表はこれで終了であるが,秋葉原での企業発表や協力企業様への報告会といった発表の機会が私たちにはまだ残っているので今の状態で止まることなく,常に上を目指しながら最後まで活動をする所存である.
\bunseki{藤原 由美恵(未来大)}
