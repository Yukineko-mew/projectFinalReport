\subsection{坂本 豊教}
\par
活動内容と予定
\par
  5月ではWikiリーダになり,開発班はサーバ班に所属を決めた.サーバ班でWikiを2014年度版にするときに,
前年度のサーバリーダに助けていただき,移行することが出来た.
また,プロジェクトで開発するアプリケーションのアイディア出しを行った.
6月は第一回合同合宿があり,開発するアプリケーションを決定した.その後OSSセミナーに参加して,サーバサイド技術について学んだ.
そしてアプリケーションの仕様書作成が始まった.
7月は中間報告書の作成や,中間発表用ポスター・スライド作成を始めた.
実際にサーバを構築することを再び助けていただきつつ,完了することが出来た.
8月から9月はもう1つのアプリケーションサーバを構築し,長崎大との連携をしつつ,技術習得を引き続き行った.
10月にはRedmine導入やssh転送先の特定を試みるも失敗した.
キャンパスベンチャーグランプリにビジネスモデル作成のサポートをし参加した.
11月はアカデミックリンクと第二回合同合宿に参加した.そして,最終発表用スライド・台本作成のサポートを行った.
12月は最終発表に参加し,最終報告書作成を今後も行っていく.
1月2月はアプリケーションの動作テスト,仕様書の最終見直し,協力企業に提出するDVDコンテンツ集めを予定している.
秋葉原課外発表会,企業報告会を2月初めに行う.

\par
5月
\begin{itemize}
\item Wikiリーダに就任
\item アプリケーション案の発案
\item サーバ班に参加,PHPの勉強
\item 3大学間で使うWikiを2014年度版に移行
\item 会議やブラッシュアップでのアプリケーション案の絞り込み
\item グループで会議,スライド作成
\end{itemize}
6月
\begin{itemize}
\item 第一回合同合宿
\item サーバ班新リーダ就任
\item OSSセミナーに参加
\item サーバとアプリケーション間で通信をするための技術習得
\par  HTTPメソッドであるPOST,GETについて勉強
\item グループでの「Rhyth/Walk」の検討
\item 「Rhyth/Walk」要求定義書,要件定義書,サービス仕様書の作成
\item 「Rhyth/Walk」サービス仕様書リーダに就任
\end{itemize}
7月
\begin{itemize}
\item 中間報告書,詳細仕様書の作成
\item 中間発表会ポスター,スライドの作成
\item 「Rhyth/Walk」サーバの構築
\item 新美先生に停電時のサーバが行う作業や、サーバの技術習得についての相談
\item 中間発表会
\end{itemize}
8月
\begin{itemize}
\item 「Cool Japanimation」サーバの構築
\item PHP・MySQLの技術習得・検証
\item Wikiのバックアップ
\end{itemize}
9月
\begin{itemize}
\item PHP・MySQLの技術習得・検証
\item クライアントのJava,iOS,HTMLの勉強
\item アプリケーションとサーバ間通信の勉強
\item 成果発表会の準備
\end{itemize}
10月
\begin{itemize}
\item PHP・MySQLの技術習得・検証
\item RedMineをアプリケーションサーバに導入,PHPと共存出来ずRedMineアンインストール
\item プロキシのssh転送について調査
\item JavaとPHPのPOST通信に関する調査
\item SwiftとPHPのPOST型送受信に関する調査
\item キャンパスベンチャーグランプリに参加
\item 成果発表会の準備
\end{itemize}
11月
\begin{itemize}
\item アカデミックリンクに参加
\item 第二回合同合宿に向け,スライド,作成
\item 第二回合同合宿に参加
\item 成果発表会で使用するスライド,台本作成のサポート
\item 最終報告書の準備
\end{itemize}
12月
\begin{itemize}
\item 最終報告書の準備
\item 成果発表会
\item アプリケーションのロゴ確定
\item Wikiのバックアップの予定
\end{itemize}
1月
\begin{itemize}
\item アプリケーションの動作テストをする予定
\item 仕様書の最終見直しの予定
\item 協力企業に提出するDVDコンテンツ集めの予定
\item Wikiのバックアップの予定
\end{itemize}
2月
\begin{itemize}
\item 秋葉原課外成果発表会
\item 企業報告会
\end{itemize}

\par
考察
\par
  私は本プロジェクトで初めてサーバを触れることになり,何をするものか,具体的にどうプログラミングするのかなど何もわからなかった.
はじめに前年度のサーバ担当者にサーバのアカウントとサーバを操作するための鍵を作ってもらった.
仕組みはわからなかったが,講義で説明を受けたり,参考書やインターネットでキーワードを検索することによって,
仕組みを勉強した.
全てが初めてで調査ばかりであったが,OSSセミナーなどの講義を受けることでサーバの知識を得ていった.
サーバにアクセスできるようになり,トライ・アンド・エラーを重ねることで,サーバの操作に慣れた.
サーバ班の最初のタスクは,2014年度版のWikiを作成することだった.
先輩に手伝ってもらいながら,2014年度版Wikiを作成した.次に行ったことは,WikiのURLを変更したことである.
IPアドレスとドメイン名を結びつけることや,Webサーバについて学ぶことができた.
先輩にサーバ室へ連れて行ってもらい,サーバに関する話を聞いたり,作業を行なった.
わからないことは,OB/OGや,ミライケータイプロジェクトの担当教員に聞き,問題を解決した.
アプリケーションが決まり,アプリケーションの実装段階になり,何のツールを使い,
何をすべきかわからなくなってしまったので,アドバイザーである新美先生に何度も相談にのってもらった.
相談することで得たキーワードを元にApache,LINUX,MySQLを使ってWebサーバを構築した.
サーバ班のメンバとデータベースを設計し,データベースサーバを構築した.
PHPを使って,データベースを操作できるようにサーバサイドプログラミングを行なった.
日々の活動の中では,メンバ間で交代し議事録を取った.
わかりやすい議事録を取るのを心がけ,時にはメンバ間での知識の相違を埋める資料作成も行った.
\par
  中間発表会では,発表スライドの作成を行った.
スライドを作成し先生にレビューをしてもらい,修正を加え,その工程を数回繰り返し,スライドを作成した.
最終発表にはスライド作成メンバとスライド,台本を作成した.
スライドを教授らから評価をもらい,何度も直し,作成した.
サーバに関しては,サーバ通信を主に携わってきた.iOS端末は持っておらず,Swiftも利用しているのでデバックがしづらいがなんとか
やっていこうと努力している.
まだまだ知識が不十分であると感じているので,これからも調査してプログラミングを行い,
クライアント側の動作も理解し勉強していきたいと感じた.
また,本プロジェクトの最終ゴールである企業報告会がある.
これからは,企業報告会に向けスライドや,ポスターなどの発表資料や,アプリケーション自体の改良を行なっていく予定である.
\par
\bunseki{坂本 豊教(未来大)}
