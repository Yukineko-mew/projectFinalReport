\subsection{Android班}
\par この節では,本プロジェクトのアプリケーションである「Cool Japanimation」を開発するために行ったプロセスについて述べる.
\par 以下,前期のプロセスを述べる
\begin{itemize}
\item Androidアプリケーション開発のための環境構築
\item Androidプログラミングおよび開発工程を効率よく学ぶため,参考書のサンプルコードを参考に簡単なアプリケーションを作成.
\item 画面遷移図の完成
\item Androidプログラミングの基本を勉強
\item Eclipseの基本操作の勉強
\end{itemize}

\par
前期の本アプリケーションのAndroid班の開発プロセスは,まず環境の構築をした.そのために,Javaの開発環境としてEclipseをまた,Androidの開発に必要なSDKなどをインストールした.また,まずAndroid開発になれるために,参考書をもとに簡単なAndroidアプリケーションを開発して,Android開発の基本を学んだ.

\par 
以下,後期のプロセスを述べる
\begin{itemize}
\item 開発スケジュールの決定
\item 決定した画面遷移図をもとに画面の開発.
\item 画面を実際に遷移させる
\item 全ての機能の画面をマージ
\item データの内部保存の実現
\item サーバとの連携に向けた開発
\item 未実装機能の実装スケジュール(優先度)を決定
\item 未実装機能の実装
\end{itemize}

\par
後期の開発プロセスとしてはまず,本格的に開発スケジュールを決定し,仕様書で決定した画面遷移図をもとに,機能ごとに画面を開発し,その画面が遷移するようにした.その後,一度全ての機能をマージし,画面遷移のみのアプリケーションを開発した.そして,各データを内部保存できるよう実装し,順次,サーバに保存できるように実装していった.まだ完全に実装できていない機能については,優先度を決定しその優先度どおりに実装していった.

\bunseki{紺井 和人(未来大)}
