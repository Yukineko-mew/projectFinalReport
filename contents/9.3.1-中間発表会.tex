\subsection{オープンキャンパス}
\par オープンキャンパスに向けてはプロジェクト説明とアプリケーション説明のための2つのポスターを作成した.プロジェクト説明ではプロジェクトの活動内容や目標,スケジュールを来場者に紹介した.アプリケーション説明ではアプリケーションの機能を紹介した.また,実機デモの展示も行った.これはサービス仕様書を基に選曲アルゴリズムを除いた,ランダム音楽再生,および画面遷移を行うAndroidアプリケーションを作成した.
\bunseki{遠藤 崇(神奈工)}

\subsection{幾徳祭}
\par 幾徳祭に向けてはプロジェクト説明と進行状況説明のためのポスターを作成した.プロジェクト説明ではオープンキャンパスと同様の内容で説明し同様の内容で説明した.進行状況説明ではアプリケーションの機能,企画段階での動き,開発状況を説明した.実機デモは実際に開発中のアプリケーションを用いた.
\par 来場者からは今後の開発に関する助言をいくつかもらうことができた.具体的には解析機能の精度向上方法や早期完成のための機能削減の提案などをもらった.
\bunseki{遠藤 崇(神奈工)}
