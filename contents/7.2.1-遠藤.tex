\subsection{遠藤 崇}
\par 私は神奈工リーダに就き,セミナー,学内ミーティングの進行,担当の振り分け,各作業のフォローなどを行った.発表資料作成,ドキュメント作成,アプリケーション開発にも携わり,開発ではRhyth/WalkのBPM解析機能を担当した.
\par 以下に作業内容を列挙する.

4月
\begin{itemize}
\item セミナーに参加
\item 神奈工リーダに就任
\end{itemize}
5月
\begin{itemize}
\item Androidアプリケーション制作のための技術習得
\item アプリケーションのアイディア出し
\item アイディア提案シートの作成
\item セミナー外学内ミーティングのセッティング
\item 第一回合同合宿の資料作成の振り分け
\item 第一回合同合宿の資料作成のフォロー
\item 技術習得用デモ作成
\item リーダ会議
\end{itemize}
6月
\begin{itemize}
\item 第一回合同合宿
\item 合宿グループワークでの記録係
\item 要求定義書の作成
\item 要件定義書の作成
\item 画面遷移図の作成
\item ビジネスモデル説明の作成
\item 類似アプリケーション調査
\end{itemize}
7月
\begin{itemize}
\item サービス仕様書の作成
\item 中間報告書の作成
\item オープンキャンパス展示の準備
\item オープンキャンパス展示
\end{itemize}
8月~10月
\begin{itemize}
\item 必要な技術を習得
\item アプリケーション開発
\end{itemize}
11月
\begin{itemize}
\item 第二回合同合宿
\item アプリケーション開発
\item 幾徳祭展示の準備
\item 幾徳祭展示
\end{itemize}
12月
\begin{itemize}
\item 学内最終発表の準備
\item 学内最終発表
\item 最終報告書の作成
\end{itemize}
1月
\begin{itemize}
\item 最終報告書の作成
\item 最終報告会
\end{itemize}
2月
\begin{itemize}
\item 企業報告会の準備
\item 企業報告会
\end{itemize}

\par リーダの主な作業である進行は慣れた作業ではなく,手探りな場面が多かった.最初の企画でのアイデア出しでは段取りが悪くまた準備が悪かったため,アイデアの質に差が出たり,拡張に関する議論が少なかった.しかし次第に様式を持ったシートに書き起こす,必要なことを列挙するなどをすることで,何をすれば良いかが分かるようになった.また,最近では発表会の2週間程度前に学内ミーティングを行い,スケジュールおよび担当者を明確にするようにしている.これにより以前よりも迅速な作業ができるようになった.
\par 開発にも力を入れた.担当部分では解析方法として2案を提案し,それぞれ実装も行った.採用はアプリケーションの開発メンバとの会議で決定した.コードは精度向上に柔軟に対応できるよう,解析に用いるパラメータを用意した.また神奈工メンバに対して状況を確認しながら助言やベースとなるコードの準備を行った.未来大を含む開発メンバに対してはコード管理をGithubで行う具体的案を作成,提案した.
\par 反省点としては発表会の準備不足がある.スケジュールには発表練習の期間が設けられたが,実際にはスライド資料のブラッシュアップに費やされた.結果としてベストな発表ができなかった.スケジュールの後半に作業が詰まっているためと考える.準備の着手を3日から1週間程度早めにするべきだっただろう.

※\bunseki{遠藤 崇(神奈工)}
