\subsection{第2回合同合宿}
\par
概要
\par 
第二回合同合宿は,11月23日,11月24日の2日間に渡り,未来大にて行った.本項では第二回合同合宿の日程,成果,反省についてまとめる.
\par
日程
\par
【会議第1部】(23日午前)
\begin{itemize}
\item 全体の進捗確認
\item 各大学ごとの進捗確認
\end{itemize}
【会議第2部】(23日午後)
\begin{itemize}
\item 各アプリケーションのデモ発表
\item グループに分かれてデモシナリオの作成,機能の優先順位の決定
\item 上記話し合いの報告
\end{itemize}
【会議第3部】(24日)
\begin{itemize}
\item デモシナリオの発表,レビュー
\item 26日の合宿会議に向けてのインプットの話し合い
\item 上記話し合いの報告
\end{itemize}

\par
良かった点・学んだ点
\begin{itemize}
\item 進捗の遅れを見直す良い機会となった.
\item プロジェクトと自分を客観的に見つめ直すことが出来た.
\item 企業の方からや OB・OG の方からドキュメントやアプリケーションについてのコメントを頂けたことも今後のためになったと感じた.
\item 現状を再確認し,危機感を持つ事が出来た.
\item アプリケーションについて何を伝えれば魅力を伝えられるかが見えてきた.
\item 
\end{itemize}
\par
悪かった点・改善したい点
\begin{itemize}
\item 進捗報告発表資料やグループワークの準備不足が多く見られた.
\item 前の発表者がもらったコメントに対応することができなかった.
\end{itemize}
\par
まとめ
\par 今回は大幅なスケジュールの変更があり臨機応変に対応することが難しかったが,変更したことで各アプリケーションとも有意義な話し合いができたのではないかと思う.

\bunseki{安藤 歩美(神奈工)}
