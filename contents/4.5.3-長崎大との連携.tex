\subsection{長崎大学との連携}
 今回のプロジェクトから長崎大が加わったため,二つの大学で共同開発を行うことになった.話し合いや連絡事項を伝える方法として,最初はSkypeを使用して行っていたが,メンバの都合を合わせるため夜に行うことが多く,長引いた場合深夜までかかることがあったり,メンバ全員がきちんと参加できているかの確認ができなかったりなどの問題点が多かったため,LINEを使用して行うこととなった.LINEはメンバ全員が頻繁に使用しており,スマートフォンでいつでも起動できるため,話し合いの時間を合わせやくなり,深夜までかかることがなくなった.また,全員が連絡事項を確認できたかは,既読数を確認することで把握することができた.
 第一回合宿に長崎大の途中から参加したメンバが参加できなかったため,アプリケーションの概要や合宿の内容を共有することが難しかった.また,第二回合宿では,長崎大がSkypeでの参加になったため,話のやり取りを行うことが難しかった.
 共同開発では,未来大と長崎大とで機能の認識の相違があったため,合わせることが大変だった.仕様書など共同で編集するものはGoogleDriveを使用して,いつでも編集できるような環境を整えた.また,お互いの進捗状況の確認はWikiを使用した.
\bunseki{大鶴宗慶(長崎大)}
