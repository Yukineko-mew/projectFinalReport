\subsection{赤木 詠滋}
アプリケーションの開発ではAndroid向け「Rhyth/Walk」の開発を担当しており,
個人の作業として,主に必要書類の作成や,音楽解析のスケール検出を担当していた.
他にも,アプリケーション決定以前には、神奈工セミナにおいて複数のアプリケーション案を提供行い,
それをセミナメンに提案するために資料の作成なども行った.
また,アプリケーション案を第一回合同合宿で提案するために,アプリケーション案のマージ案を提案し,必要資料の作成を行った.
スケール解析では,波形データの周波数成分をFFTを用いて検出し,
そのデータを用いて音階を判定するための条件式の構築などを行っていた.\par
4月
\begin{itemize}
\item セミナーに参加
\end{itemize}
5 月
\begin{itemize}
\item Androidアプリケーション製作のための技術習得
\item アプリケーションのアイディア出し
\item アイディアシートの作成
\item 第一回合同合宿の資料作成のフォロー
\item 技術習得報告デモの作成
\end{itemize}
6 月
\begin{itemize}
\item 第一回合同合宿
\item 合宿グループワークでのビジネスモデル係
\item 要求定義書リーダ会議
\item 要件定義書リーダ会議
\item 要件定義書の作成
\item 類似アプリケーション調査
\end{itemize}
7 月
\begin{itemize}
\item サービス仕様書のための会議に参加
\item 中間報告書の作成
\item 中間発表
\end{itemize}
8 月
\begin{itemize}
\item 必要な技術を習得
\item アプリケーション開発
\end{itemize}
9 月
\begin{itemize}
\item 必要な技術を習得
\item アプリケーション開発
\end{itemize}
10 月
\begin{itemize}
\item 必要な技術を習得
\item アプリケーション開発
\end{itemize}
11 月
\begin{itemize}
\item 第二回合同合宿
\item アプリケーション開発
\end{itemize}
12 月
\begin{itemize}
\item 学内最終発表準備
\item 学内最終発表
\item プロジェクト最終報告
\item 最終報告書の作成
\end{itemize}
1 月
\begin{itemize}
\item 最終報告書の作成
\item 最終報告会
\end{itemize}
2 月
\begin{itemize}
\item 企業報告会の準備
\item 企業報告会
\end{itemize}
個人作業として特筆すべき点は,最初に書いたが「スケール解析」の部分が挙げられる.
実はアプリケーション開発開始時にはこの項目は無く,「どのようにして音楽を解析するか,シチュエーションの要素を割り振るか」という事から考え始めていた.
また,スケール解析のことを考え付いた後にも,どのようにして判定するか,判定式はどうするか,のばらつきはどうするのかなど,非常に頭を悩ませさせれた.
そのため現段階では,まだ「全ての楽曲データからスケールを解析できるようになる」という完成には程遠い完成度だ.
しかし,ピアノソロの曲ではかなり正確なスケールの検出が可能なレベルまで持ってこられている.
また,まだ検討段階ではあるが,私が担当しているスケール解析ではそれを元にして,人の感情との関連付けを行えないかということを考えている.
これは楽曲の波形データを取り扱う部門でもまだまだ研究段階のものであり,私が行っていたことは卒業研究とほぼ同定義であったらしい.
反省点としては,アプリケーション案発表の際や最終報告の際に緊張しすぎてしまい,早口かつ支離滅裂な内容を口走ってしまったことや,けんか腰になってしまったこと.
技術力が低く,結果として大きくアプリケーションの開発を遅らせてしまっていること,積極的に自分の意見を出し始めたのが第二回合同合宿近辺だったなど,
挙げきれないほどの反省点がある.特に,発表に関してはあまりにも自分が不慣れなのだと痛感させられた.
しかし、そのおかげで「まずは場数を踏み,人前での発表に慣れていく.」という目標を立てる事ができた.
社会人として人前で発表できないというのは大きな弱点であり,克服しなければならないことだ.
ミライケータイプロジェクトに参加したおかげで、社会人になる前にこのことを自覚できたことは、とても幸福なことだと考えている.\par
\bunseki{赤木 詠滋(神奈工)}
