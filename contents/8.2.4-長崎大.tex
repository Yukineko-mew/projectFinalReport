\subsection{長崎大学}
私たちの後期の成果としては,HTML5やJava scriptを習得できたこと,メンバ全員が成長したことがあげられる.前期からHTML5の習得を行い,
夏休みから後期にかけてサーバや各機能の開発を開始した.開発の過程でアカウントの認識ができない,ナビ機能でルート表示が
クリア出来ない,などなかなか思うように動かなかったことも多かった.問題が発生するたびにメンバと協力することで解決し,
開発に必要な技術に対する理解も深めることができた.また,長崎大で行われた,長崎大,富山大,新潟大の三大学での
「ものづくり・アイデア展」では,スライドやデモ機を用いてアプリケーションの紹介や説明を行うことで,プレゼンテーション力や
コミュニケーション力を身につけることができた.
\bunseki{大鶴 宗慶(長崎大)}
