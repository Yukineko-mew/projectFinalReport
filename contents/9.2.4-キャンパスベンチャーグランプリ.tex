\subsection{キャンパスベンチャーグランプリ}
\par
キャンパスベンチャーグランプリとは,大学・大学院・高等専門学校・短期大学・専門学校の学生による新事業提案を行うコンテストだ.
\par
私たちはそれに,今年開発している二つのアプリケーションを提出した.提出内容としては,「プランの具体的内容」,「プランの優位性」,「実現方法,実行時期,課題」,「市場性」,「事業採算性・収支予測」,「事業化の意思」の 6 つの項目を基準として書類を作成した.
\par
\begin{enumerate}
\item プランの具体的内容
\par
 アプリケーションを開発しようと思った背景や,それを提案しようと思った理由,具体的なサービス内容についてを書いた.
\par
\item プランの優位性
\par
 ここでは他のアプリケーションやサービスと比較して,どのようなところが新しいのか,どのようなところを特徴としているのかということを書いた.
\par
\item 実現方法,実行時期,課題
\par
 サービスを実現するに当たっての具体的な事業計画や,それに対する課題点を追求し,その解決策を書いた.
\par
\item 市場性
\par
 本アプリケーションが想定している市場を調べ,具体的にどのくらいの人がターゲットとなりえるのかを書いた.
\par
\item 事業採算・収支予測
\par
 上記の実現方法と市場性からおおよその収益,支出を予想し,それを踏まえての5カ年計画を立てた.
\end{enumerate}
\bunseki{三栖 惇(未来大)}
\par
\subsubsection{Cool Japanimation}
\par
\begin{itemize}
\item 一次審査
\par
 9月26日から10月31日の期間で提出するための書類作成を行った.今回は未来大のみで書類作成を行った.アプリケーションでどのような利益を得るか,またそのモデルからどれくらいの規模の利益を得られるか,アプリケーションを企業として運営した場合の人件費などについて考えた.収支予測を考えるに至って,未来大では知識不足だったので,作成に大いに手間取ってしまった.また,一次審査を通過することはなかった.
\par
\item 「Cool Japanimation」反省点
\par
 例年は,ビジネスモデルを考案する上で,専修大と協力しているのだが,今回は不参加で,また着手するのが遅く,他大学に依頼する時間がなく,未来大のみで作成に着手してしまった.また,収支予測などの算出などの知識をもってなかったため,一から学ぶ必要があったため,レビューをするときにもあまり完成度の高いものを評価してもらうことはできなかったのが一番の反省点である.
\end{itemize}
\bunseki{紺井 和人(未来大)}
\subsubsection{Rhyth/Walk}
\begin{itemize}
\item 一次審査
\par
 「Cool Japanimation」と同様の期間で,同じく未来大のみで書類を作成した.場所を問わず何度も会議を重ね,もらったレビューに対してどうすべきかを班の皆で話し合い,完成へと扱ぎ付けた.収支予測など分からないことは昨年の例やネット文献,先輩の助言などを頼りに調べ,分からないなりにうまく仕上げることができた.しかし,こちらも「Cool Japanimation」同様一次審査を通過することは叶わなかった.
\item 「Rhyth/Walk」反省点
\par
 ビジネスモデルへの着手の遅さが最も反省すべき点だったと思う.門外漢である未来大生が扱うものとしては,もっと早くに力を入れて取り組むべき内容であったが,それをしっかり認識できていなかった.また,今現在できているビジネスモデルの評価として,ありきたりだという言葉が多く,考察がまだ足りていないことが分かった.未だ確定していない項目もあり,しっかりと修正及び改善していくべきだと思われる.
\end{itemize}
\bunseki{三栖 惇(未来大)}
