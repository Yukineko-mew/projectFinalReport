\subsection{ガントチャートの使用}
\par
私たちはプロジェクトを進めていくのにあたり,様々なイベントや作業ごとにスケジュール立てを必ず行った.
その際に,今年は,スケジュールを視覚的に表して一目でわかるようにガントチャートを作成した.
例年もスケジュール作成は行ってはいたが,ガントチャートを用いたスケジュール確認を行っていなかった.
ガントチャートを作成するようになったのは,プロジェクトの初回時に,プロジェクトリーダがガントチャートを作成して,スケジュールを説明していたのがきっかけであった.
後にプロジェクト活動を行う際には,開発班のリーダーや仕様書リーダを決め,スケジュールを立てるときに各グループのリーダが
作成するようにしていった.
また,ガントチャートはキャンパスベンチャーグランプリやアカデミックリンクなどといったイベントがあるごとに作成するようにし,
絶対的な締め切りを意味するマイルストーンをより意識して活動を行うことができた.
スケジュールを立てた際には,計画に変更があり調整が必要なときもあったので,その都度修正を行い,その上で
プロジェクトメンバ全員で共有していた.
さらに,後期後半になると,Cool JapanimationとRhyth/walkそれぞれでWBS(作業分解図)を作成した.
これにより,開発タスクを細分化し,各タスクに担当を分担することで,
アプリケーションの完成までの計画を管理するようにもなった.
\par
ガントチャートは企業などでよく用いられていることから,今後使う機会が多くなると予測されるので,
プロジェクト学習によって事前に作成の仕方を学ぶことができた.
このことより,スケジュールを間に合わせるためには今何をしなければならないかということを,
スケジュールを逆算して考えて,プロジェクト活動を行うことができた.
\bunseki{金澤 しほり(未来大)}
