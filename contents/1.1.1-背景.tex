\section{背景}
\par 
現在,スマートフォンの普及率は増加しており,世代問わずにスマートフォンが持っていることが当たり前の時代となった.
日本でのクライアント端末別の普及率はスマートフォンが49.8%,タブレット端末が20.1%,PCが97.0%だった.
年齢別では,スマートフォンは若年層ほど所有率が高く,高齢者層は従来型携帯電話(フィーチャーフォン)の所有率が高かった.
一方,タブレットは40代が所有率のピーク(21.6%)だった.
サービス提供では,クラウドコンピューティングというコンピューティング形態が使用されており,現在世界的に普及が進んでいる.
クラウドコンピューティングとは,サーバがユーザに提供するサービスをサーバ群を意識せずに利用が可能になるコンピューテング形態のことである.
したがってサービスやアプリケーションの可能性は年々広がってきている.
\par
具体的な例として,電子メール,カメラ機能インターネット利用,アプリケーション機能,ミュージックプレイヤー機能,GPS機能,
Bluetooth,FelicaなどのICカードを利用した非接触機能,赤外線などの無線通信機能などがあげられる.
これらの機能により携帯電話は,もはや通信機器という域を超え,日常生活の様々な場面で必要不可欠なものになっている.
様々な用途がある機能の中から今回のプロジェクトではアプリケーション機能について注目する.
そして,携帯電話にアプリケーション機能が搭載されていることは標準となり,それに伴って,docomo,Y!mobileなどの各社から
様々な機能やサービスが提供されている.
また,iOS,Android,HTML5などのプラットフォームが普及し始めたことにより,さらに幅広いサービスの利用も可能になっている.
以上のような背景を踏まえ,本プロジェクトでは既存のアプリケーションにとらわれない新しい発想で
スマートフォンアプリケーションの企画と開発を行う.
\bunseki{坂本 豊教(未来大)}
