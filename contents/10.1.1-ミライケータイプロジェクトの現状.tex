\subsection{ミライケータイプロジェクトの現状}

ミライケータイプロジェクトの現状は以下のとおりとなっている.

\begin{description}

\item[仕様書の作成状況]\mbox{}\\ 
前期の活動において要求定義書,要件定義書,サービス仕様書,詳細仕様書の作成を行った.
しかし,現状では仕様書作成の際に知識と技術が不足していたために,
開発のほうが仕様書の内容よりも先行している.
したがって今後修正する必要がある.

\item[アプリケーションの開発状況]\mbox{}\\ 
「Cool Japanimation」と「Rhyth/Walk」ともに,未来大での最終報告会においてデモアプリケーションの発表を行った.
しかし,現状ではすべての機能がひととおり開発できたものの,内部の処理が完全ではなく,これから先も開発を
続けていく必要がある.

\item[アプリケーションのテスト状況]\mbox{}\\ 
現状では,アプリケーションのテストはまだ行えていない.これから,各メンバでテスト項目を洗い出し,単体テストを行った後,
担当しているアプリケーションを相互に受入テストする予定である.

\item[ビジネスモデルの作成状況]\mbox{}\\ 
キャンパスベンチャーグランプリへの提出を目標に,両アプリケーションともにビジネスモデルの作成を行った.
しかし,収益モデルはある程度の仕組みを考えることができたものの,ビジネスモデルの部分がまだまだ改良の余地がある.
したがって,今後も作成していく必要がある.

\item[秋葉原での課外成果発表会の準備状況]\mbox{}\\ 
秋葉原での課外成果発表会に使用するポスターの作成を検討している.
最終報告会で使用したポスターを原案として,改良する予定である.

\end{description}
\bunseki{小笠原 佑樹(未来大)}
