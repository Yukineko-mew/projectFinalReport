\subsection{公立はこだて未来大学}
\begin{enumerate}
\item 各プラットフォームの特徴の調査
\par 各大学の各技術班が,担当するプラットフォームの特徴調査を行った.各技術班の構成については後述の 2.4 組織形態に記載する.
\item アプリケーション開発技術の習得 
\par 各大学の各技術班が,担当するプラットフォームの特徴調査を行った.各技術班の構成については後述の 2.4 組織形態に記載する.
\item 開発するアプリケーションのアイディア企画
\par プロジェクトメンバ全員が,第一回合同合宿において,民間企業とアプリケーションアイディアに投票を行い,本プロジェクトで開発するアプリケーションを決定した.
\item ビジネスモデル作成技術の習得
\par 各大学で,ビジネスモデルキャンバスというツールを用いて,ビジネスモデルの作成技術を習得した.
\item 類似サービスの調査
\par プロジェクトメンバ全員が,第一回合同合宿で決定したアプリケーションの類似サービスが存在するか調査した.
\item 開発アプリケーション名の決定
\par プロジェクトメンバ全員で,アプリケーションの名前を考案し,投票して名前を決定した.
\item 要求定義書の作成
\par プロジェクトメンバ全員が,項目ごとに分担して作成した.
\item 要件定義書の作成
\par プロジェクトメンバ全員が,項目ごとに分担して作成した.
\item サービス仕様書の作成
\par プロジェクトメンバ全員が,項目ごとに分担して作成した.
\item 詳細仕様書の作成
\par プロジェクトメンバ全員が,項目ごとに分担して作成した.
\item アプリケーションの機能選定
\par プロジェクトメンバ全員が,実装する機能の絞り込みを行った.
\item  前期提出物の作成
\par 公立はこだて未来大学のプロジェクトメンバが,各3大学が作成した報告書をまとめた.
\end{enumerate}

●後期
\begin{enumerate}
\item サーバ処理の実装
\par 未来大と長崎大のサーバ班が担当した.
\item アプリケーションの機能の実装
\par 各大学の開発班が,項目ごとに分担して開発した.
\item ビジネスモデル作成技術の習得
\par 未来大が主に担当し,プロジェクトメンバ全員も講演などを通じて学んだ.
\item ビジネスモデルの構築
\par 未来大が担当した.
\item キャッチコピー,デモシナリオの作成
\par 各大学が連携して作成した.
\item 後期提出物の作成
\par プロジェクトメンバ全員が,項目ごとに分担して作成した.
\end{enumerate}
\bunseki{三栖 惇(未来大)}
