\subsection{神奈川工科大学}
\par 後期ではアプリの開発,および成果発表を主に行った.前期で得られた成果に加えて,開発するにあたって企画段階で必要なものの理解,共同開発の経験,プレゼンテーションの経験の2つの成果を上げることができた.
\par 神奈工メンバは音楽解析を行う部分の開発を担当しているが,その開発の最終段階を理解できていなかった.原因として各仕様書の内容不足がある.本来ならウォーターフォールモデルのように仕様を確定させてから開発を行うべきであるが,知識不足により開発しながらの仕様確定が目立った.前もってプログラムの出力するものや構造,挙動などを考える必要があることを実感した.
\par 開発は未来大との共同開発によって行われている.コードの管理にはGitHubを用い,神奈工が形式の提案を行った.マージ作業やコメントなどのコード管理が容易になり,開発を進めるうえで有利になっている.
\par プレゼンテーションは第二回合宿,幾徳祭展示,学内最終発表と機会があり,プレゼンテーションに対する抵抗が以前より少なくなった.各プレゼンテーションともに練習することが少なくベストなものではなかった.練習することの重要性も認識できた.
※\bunseki{遠藤 崇(神奈工)}
