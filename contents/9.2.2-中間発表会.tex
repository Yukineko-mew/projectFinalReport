\subsection{中間発表会}
\par
2014年7月11日,未来大学にて前期におけるプロジェクト学習の成果を発表する,中間発表会が行われた.
\par
それに向けて,6月下旬から今年作成するアプリケーション「Cool Japanimation」と「Rhyth/Walk」の発表用のデモストレーション,スライド,ポスターの発表資料それぞれを作成し始めた.
\par
以下に中間報告会で用いた各種発表資料の準備内容と役割分担を示す.
\begin{enumerate}
\item
「Cool Japanimation」 デモストレーション(Android:岩田,HTML5:金澤)
\par 中間発表会に向けて,「Cool Japanimation」のデモストレーションをAndroidとiPhoneのスマートフォン端末に実装した.実装では,サービス仕様書で作成した遷移図を利用し,アルゴリズムが必要な機能を除き,画面遷移ができるデモを実装した.
\item Rhyth-Walk デモストレーション(Android:岩田,iOS:三栖 )
\par
中間発表会に向けて,Rhyth/WalkのデモストレーションをAndroidとiPhoneのスマートフォン端末に実装した.実装では,サービス仕様書で作成した遷移図を利用し,アルゴリズムが必要な機能を除き,画面遷移ができるデモを実装した.
\item ポスター(未来大:三栖,木津,澤田)
\par
中間発表会で使用するポスタは,プロジェクトの概要を示したメインポスターを1枚,アプリケーションそれぞれの説明を示したサブポスターを2枚,合計3枚作成した.メインポスターでは本プロジェクト学習の概要,目的,運営方法,スケジュールなどを記載した.サブのポスターでは,開発する2つのアプリケーション「Cool Japanimation」と「Rhyth/Walk」の説明と主な機能についてをコンテンツとした.ポスター作成では学生,教員の評価を何回も行っており,全員が納得のいくものを作成することが出来た.
\item 発表用スライド(未来大:岩田,村上,藤原)
\par
本プロジェクト学習の概要や目的,運営方法,アプリケーションの説明,スケジュール,中間発表会時点での進行状況とそれまでの流れをスライドにしたものである.しかし,発表時間を13分としたため全てを伝えると時間が足りず,一番私達が伝えたい今年の特徴としてアプリケーションの説明に重点をおいてスライドを作った.
\item 発表者(未来大学生全員)
\par
作成されたスライド,ポスターを用いて聴衆の前で中間発表会を行った.発表回数は6回で,メンバ12人が3人1組みの6組で全員が発表を行った.スライドの内容を把握し,アドリブを加えながら発表者それぞれの言葉で行った.中間発表会前にプレゼンテーションをメンバで相互レビューし合うなど発表練習を綿密に行った.当日は多くの来場者に本プロジェクトの活動と成果を伝えることができた.
\item 評価シート(未来大:紺井)
\par
評価シートはWGのテンプレートではアプリケーションについての評価を得ることができなかったので,裏面にアプリケーションについて評価していただけるように新しくそれぞれのコメント欄を設けた.
\par 評価シートの内容は発表技術,発表内容,「Cool Japanimation」について,「Rhyth/Walk」についてと10段階で評価してもらい,それぞれの評価基準はプロジェクトの内容を伝えるために,効果的な発表が行われているか,プロジェクトの目標設定と計画は十分なものであるか,各アプリケーションケーションそれぞれに使いたいと思えるアプリケーションだったかとした.また点数だけでなくアドバイスや意見を多く書けるようにコメント欄を大きくとった.なお10段階評価は1が悪く10が良いとなっている.評価シートには表面の右上に1から6までの番号をあらかじめ記入し,どの発表者の評価なのかわかるようにした.
\end{enumerate} 
\par
全体評価
\par
中間発表の聴講者によるアンケート調査の結果を踏まえて自分のグループの評価を行うと,発表技術についてはスライドは「内容が頭に入りやすい」などの高評価が得られたが,発表者については「聞きとりやすい」というものや「わかりやすい」などという高評価なものがあれば,「声が聞きとりにくい」というものや「練習不足がみられた」という評価も頂いた.アプリに関して「はCool Japanimation」はFacebookなどのSNSとの差が明確でなく,また「Rhyth/Walk」では定義されている気分ではユーザの気分と一致しないのでないなどコメントがあった.しかし、これからの実装を心待ちにしている人も多数いた.
\bunseki{中司 智朱希(未来大)}
