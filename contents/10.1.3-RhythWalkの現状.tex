\subsection{Rhyth/Walkの現状}
「Rhyth/Walk」の現状は以下のとおりとなっている.

\begin{description}

\item[仕様書の作成状況]\mbox{}\\ 
前期の活動において要求定義書,要件定義書,サービス仕様書,詳細仕様書の作成を行った.
アプリ名が変更されていたり,詳細にビジネスモデルが書いていないなどの不備がある.

\item[アプリケーションの開発状況]\mbox{}\\  
AndroidとiOSの2つのプラットフォームで開発を行っているが,未完成の機能がある.また実装済みの機能の中にも精度を向上する必要のある機能が存在する.

\item[ビジネスモデルの作成状況]\mbox{}\\ 
11月のキャンパスベンチャーグランプリへの提出を目標に,ビジネスモデルの作成を行った.
これにより,ビジネスモデルを考えていくにしたがっての,根幹を作成することができた.
しかし,まだまだ既存のビジネスモデルにあやかるばかりで,「Rhyth/Walk」としての独自性がない.

\item[サーバ連携]\mbox{}\\
「Rhyth/Walk」はサーバ連携が不十分である.iOSではサーバとの通信を行えていない.

\item[学外発表会の準備状況]\mbox{}\\ 
秋葉原での課外成果発表会に使用するポスターの作成を検討している.最終報告会で使用したポスターを原案として,改良する予定である.
さらに,秋葉原での課外成果発表会では,アプリケーションの概要のポスターだけではなく,ビジネスモデル専用のポスターを作成する予定である.
\per
加えて,協力してくれた企業への,企業報告会も行う予定である.そこでは,今年一年の成果をスライドによるプレゼン形式で発表したり,デモンストレーションを作成して,実際に「Rhyth/Walk」がどのようなアプリケーションになったのかをみてもらう予定である.

\end{description}
\bunseki{村上 惇(未来大)}
