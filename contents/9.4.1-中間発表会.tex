\subsection{中間発表会} 
\par
2014年11月14日,長崎大学にて成果発表会に向けて成果を発表する,中間発表会が行われた.
それに向けて,11月上旬から発表するアプリケーション「Cool Japanimation」の発表用のスライドをそれぞれ作成し始めた.
以下に,中間発表会で用いた各種発表資料の準備内容と役割分担を示す.
\begin{enumerate}
\item 「Cool Japanimation」 
\par
\item スライド(アプリケーションの説明・サーバ:磯野,アカウント:岡本,ナビ:吉澤) 
\par
本プロジェクトにおいての,アプリケーション案の洗い出しやコンセプト,ターゲット,アプリケーションの説明,ビジネスモデル,現在の進捗状況,今後の予定,そして画面遷移が行われている映像をスライドにしたものである.
アプリケーションの概要説明のスライドは,第一回合同合宿のアプリケーションを説明する際に使ったスライドを元に開発を行い,各機能の説明はその機能の担当者が作成を行った.
発表を聞いてもらったのがアプリケーション開発は専門外という方々が多かったため,機能の技術を説明するのは最小限にして,アプリケーションの特徴を説明することを重視した.
また,今回の中間発表会が比較的大人数に向けたものであったのと,時間が10分と限られていたため,実際に実機に触れてもらうデモンストレーションを行うのは非効率であると考えた.
そこで,実際に画面遷移が行われている映像を見せながら説明を行うようにすることにより,より多くの人にアプリケーションの特徴を知ってもらうことができた.
\item 発表者(長崎大:磯野,岡本,吉澤)
\par
作成されたスライドを用いて聴衆の前で中間発表会を行った.
発表会では私たちの他に3つのチームが発表した.
1チームの発表時間は10分とされ,その後質疑応答の時間が設けられた.
発表者は担当のスライドの内容を把握し,損門的分野ではなくてもわかるような発表を心がけた.
中間報告前にスライドをメンバ全員で確認し,意見を出しあった後,先生にも見てもらい,より良い報告ができるように務めた.
当日は多くの来場者に本プロジェクトの活動と成果を伝えることができた. 
\item 質疑応答(長崎大:磯野) 
\par
質疑応答の時間では,15分ほどの時間が取られ,他のチームのメンバや,偏った分野ではなく,様々な分野の先生方から発表やアプリケーションの機能に対する意見をもらった. 
\item 評価 
\par
中間発表の聴講者による質疑応答から自分のグループの評価を行うと,発表技術についてはスライドに関して,発表順序を変えたほうがいいのではといった意見や,聖地や巡礼といった専門的な用語の説明が不十分ではないかという意見をもらった.
アプリケーションに関しては,もっとこの角度から写真を撮れる同じ場面が撮れるといった細かい要望や,果たしてこの機能をアプリケーションでするメリットとは何なのか,といった厳しい意見をもらった.
偏った分野ではなく,様々な分野の先生方から意見をもらうことができて,私たちでは思いつきもしなかったアイデアや問題があり,とても有意義な時間となった.
\end{enumerate} 
\par
\bunseki{吉澤 健太(長崎大)}
