\subsection{アプリケーション案の決定}

\par
本プロジェクトで開発することになるアプリケーションの決定は,第一回合同合宿の活動として行った.
1日目には,各大学が持ち寄った合計6個のアプリケーションのアイディアを発表して共有を行った.
発表は各大学のプロジェクトメンバだけでなく,各大学の担当教員,企業の人々やミライケータイプロジェクトの
OBも質疑応答に参加することで,様々な視点から各アプリケーションのアイディアに対して理解を深めたり,
改善点を発見したりといった評価をすることができた.

\par
すべてのアイディアを発表し終わった後,持ち寄った6個のアイディアを1分間の最終プレゼンテーションを行った.
その後,合宿に参加した全員の投票によって4つのアイディアに絞った.その4つのアプリケーション案に関してでき
るだけ複数の大学のメンバが割り振られるようにグループを4つ作成し,それぞれのアイディアについて洗練を行った.
同時に合宿の2日目に行う発表のための発表資料の作成も行った.
アイディアの洗練については合宿以前に使用していたアイディア提案シートに最終評価項目が追加されたものを使用して行い,
ビジネスモデルの考案についてはビジネスモデルキャンバスの項目を埋める作業を行った.

\par
2日目も1日目と同様に,グループ毎に洗練を行ったアイディアを発表し,質疑応答を行った.
また,2日目では質疑応答と同時にアイディア提案シートの項目に従ってアイディアを5段階評価し,
この評価結果を元に本プロジェクトで作成するアプリケーションを2個に決定した.
決定したアプリケーションは,外国人オタクを日本に誘致するマッチングアプリである「Cool Japanimation」と,
ユーザの周辺環境に一致するイメージを持つ楽曲を自動で選んで再生する「Rhyth/Walk」というアプリケーションであった.
\bunseki{小笠原 佑樹(未来大)}
