\subsection{類似アプリケーションの調査}

\par 
 「Rhyth/Walk」について各機能やサービスに類似するアプリケーションについて調査を行なった.
類似アプリケーションについて以下の5つの点を調査し,まとめたものをWikiにアップした.
\begin{itemize}
\item 機能
\item 類似点
\item 差別化ポイント
\item 収益モデル
\item ターゲット層
\end{itemize}
\par
以下にアップした類似アプリケーションについて記述する.
\par
1.Groove
\par
「機能」
\begin{itemize}
\item 音楽プレーヤー
\item 自動プレイリスト作成
\item 曲情報を自動で取得
\item ミックス作成
\item アカウント作成
\item 拡張機能(アドオン)
\end{itemize}

「類似点」
\begin{itemize}
\item 音楽プレーヤー
\item 音楽をリスト化
\end{itemize}

「差別化ポイント」
\begin{itemize}
\item シチュエーション自動取得
\item シチュエーションごとの優先度設定
\end{itemize}

「収益モデル」
\begin{itemize}
\item 拡張パック(170円)
\end{itemize}

「ターゲット層」
\begin{itemize}
\item iOS
\item 音楽を多数端末に入れている人
\end{itemize}
「webページタイトル」
\par
URL : http://www.groovemusicapp.com/
\\
\par
2.HummingWay
\par
「機能」
\begin{itemize}
\item 音楽プレーヤー
\item facebookアカウント登録
\item 場所に音楽を登録
\item 音楽自動再生(ON/OFF有り)
\item 検索(場所,音楽,アーティスト,ユーザ)
\item GPS連動
\item 地図
\end{itemize}

「類似点」
\begin{itemize}
\item 音楽プレーヤー
\item 特定の場所で音楽を流す
\end{itemize}

「差別化ポイント」
\begin{itemize}
\item 場所以外のシチュエーション(天気,季節,時間,歩くテンポ)
\item シチュエーションごとの優先度設定
\end{itemize}

「収益モデル」
\begin{itemize}
\item 価格無料,広告がスクリーンショットでは表示されていない.
\end{itemize}

「ターゲット層」
\begin{itemize}
\item iOS
\item 散歩,ドライブをしながら音楽を聴く人
\end{itemize}
「webページタイトル」
\par
URL : http://hummingway.at/
\\
\par
3.music Chef
\par
「機能」
\begin{itemize}
\item 音楽プレーヤー
\item シチュエーション設定(手動)
\item SNSシェア機能
\item Like/Dislike設定
\item 特集(お気に入り検索)
\item シード再生
\item シェフ選択
\item カーナビ対応
\end{itemize}

「類似点」
\begin{itemize}
\item 音楽プレーヤー
\item シチュエーションと音楽の一致
\end{itemize}

「差別化ポイント」
\begin{itemize}
\item 2015年1月31日でサービス終了
\item シチュエーションの違い(BPM)
\item 優先度設定
\item シチュエーション自動取得
\end{itemize}

「収益モデル」
\begin{itemize}
\item 1ヶ月無料
\item 月額コース(400円,900円)
\end{itemize}

「ターゲット層」
\begin{itemize}
\item iOS
\item 好みの曲を見つけたい人
\end{itemize}

「webページタイトル」
\par
URL : http://musicchef.jp/
\\
\par
4.ON-GAKU
\par
「機能」
\begin{itemize}
\item 音楽プレーヤー
\item MYDJ再生機能(楽曲判別)
\item 新譜紹介
\item Walking with MYDJ(ウォーキングに最適な曲を紹介)
\item ジャケット写真の自動表示
\item お薦め曲紹介
\end{itemize}

「類似点」
\begin{itemize}
\item 音楽プレーヤー
\item シチュエーションと音楽の一致
\item 歩いている時の選曲
\end{itemize}

「差別化ポイント」
\begin{itemize}
\item BPMによる選曲
\item ジャケットは表示しない
\item シチュエーション自動取得
\end{itemize}

「収益モデル」
\begin{itemize}
\item iTunesより曲をダウンロード(新譜紹介,お薦め曲紹介)
\end{itemize}

「ターゲット層」
\begin{itemize}
\item iOS
\item 音楽を自動選曲してもらいたい人
\end{itemize}
「webページタイトル」
\par
URL : http://on-gaku.aim-inc.info/index.html
\\
\par
5.Runtastic Pedometer
\par
「機能」
\begin{itemize}
\item 歩数計
\item 音楽アプリの呼び出し
\item どこでも計測可能:ズボンやジャケットのポケット
\item スピード測定
\item 歩いた総距離
\item 歩数頻度測定
\item SNSシェア機能
\item runtastic.comのフィットネスサイトでのデータ管理
\item 体重,身長のデータ保存によるカロリー計算
\item ユーザ登録
\item Google+ ログイン
\item 省エネモード
\end{itemize}

「類似点」
\begin{itemize}
\item 歩数計測によるテンポ検出
\item テンポ検出と音楽聴取の同時作業
\end{itemize}

「差別化ポイント」
\begin{itemize}
\item 別のアプリを呼び出さない
\item シチュエーション取得
\item 自動選曲
\end{itemize}

「収益モデル」
\par
「ターゲット層」
\begin{itemize}
\item iOS,Android
\item 健康管理をサポートして欲しい人
\end{itemize}
「webページタイトル」
\par
URL : https://www.runtastic.com/ja/apps/pedometer
\\
\par
6.Splyce
\par
「機能」
\begin{itemize}
\item 音楽プレーヤー
\item プレイリスト
\item 音楽のテンポ検出
\item 音楽のテンポアレンジ
\item 音楽の順番変更
\item 画面の明るさ変化
\item カメラ用LED点滅
\end{itemize}

「類似点」
\begin{itemize}
\item 音楽プレーヤー
\item 曲のBPM検出,選曲に利用
\end{itemize}

「差別化ポイント」
\begin{itemize}
\item BPMと歩いているテンポを選曲に利用
\item シチュエーション取得
\item 優先度設定
\end{itemize}

「収益モデル」
\begin{itemize}
\item Full feature set(200円)
\item Fancier Upgrade(100円)
\item Pro feature set(100円)
\end{itemize}

「ターゲット層」
\begin{itemize}
\item iOS
\item DJプレイをしてみたい人
\end{itemize}
「webページタイトル」
\par
URL : http://inqbarna.com/mobile-apps/splyce.html

\par
\bunseki{坂本 豊教(未来大)}
