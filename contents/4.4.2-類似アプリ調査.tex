\subsection{類似アプリケーションの調査}


\par 
 「Rhyth/Walk」について各機能やサービスに類似するアプリケーションについて調査を行なった.
類似アプリケーションについて以下の5つの点を調査し,まとめたものをWikiにアップした.
\par
1.機能
\par
2.類似点
\par
3.差別化ポイント
\par
4.収益モデル
\par
5.ターゲット層
\par
以下にアップした類似アプリケーションについて記述する.
\\

\par
1.Groove
\par
「機能」
\par
音楽プレーヤー
\par
自動プレイリスト作成
\par
曲情報を自動で取得
\par
ミックス作成
\par
アカウント作成
\par
拡張機能(アドオン)
\par
「類似点」
\par
音楽プレーヤー
\par
音楽をリスト化
\par
「差別化ポイント」
\par
シチュエーション自動取得
\par
シチュエーションごとの優先度設定
\par
「収益モデル」
\par
拡張パック(170円)
\par
「ターゲット層」
\par
iOS
\par
音楽を多数端末に入れている人
\par
URL
\par
http://www.groovemusicapp.com/
\\

\par
2.HummingWay
\par
「機能」
\par
音楽プレーヤー
\par
facebookアカウント登録
\par
場所に音楽を登録
\par
音楽自動再生(ON/OFF有り)
\par
検索(場所,音楽,アーティスト,ユーザ)
\par
GPS連動
\par
地図
\par
「類似点」
\par
音楽プレーヤー
\par
特定の場所で音楽を流す
\par
「差別化ポイント」
\par
場所以外のシチュエーション(天気,季節,時間,歩くテンポ)
\par
シチュエーションごとの優先度設定
\par
「収益モデル」
\par
価格無料,広告がスクリーンショットでは表示されていない.
\par
「ターゲット層」
\par
iOS
\par
散歩,ドライブをしながら音楽を聴く人
\par
URL
\par
http://hummingway.at/
\\

\par
3.music Chef
\par
「機能」
\par
音楽プレーヤー
\par
シチュエーション設定(手動)
\par
SNSシェア機能
\par
Like/Dislike設定
\par
特集(お気に入り検索)
\par
シード再生
\par
シェフ選択
\par
カーナビ対応
\par
「類似点」
\par
音楽プレーヤー
\par
シチュエーションと音楽の一致
\par
「差別化ポイント」
\par
2015年1月31日でサービス終了
\par
シチュエーションの違い(BPM)
\par
優先度設定
\par
シチュエーション自動取得
\par
「収益モデル」
\par
1ヶ月無料
\par
月額コース(400円,900円)
\par
「ターゲット層」
\par
iOS
\par
好みの曲を見つけたい人
\par
URL
\par
http://musicchef.jp/
\\

\par
4.ON-GAKU
\par
「機能」
\par
音楽プレーヤー
\par
MYDJ再生機能(楽曲判別)
\par
新譜紹介
\par
Walking with MYDJ(ウォーキングに最適な曲を紹介)
\par
ジャケット写真の自動表示
\par
お薦め曲紹介
\par
「類似点」
\par
音楽プレーヤー
\par
シチュエーションと音楽の一致
\par
歩いている時の選曲
\par
「差別化ポイント」
\par
BPMによる選曲
\par
ジャケットは表示しない
\par
シチュエーション自動取得
\par
「収益モデル」
\par
iTunesより曲をダウンロード(新譜紹介,お薦め曲紹介)
\par
「ターゲット層」
\par
iOS
\par
音楽を自動選曲してもらいたい人
\par
URL
\par
http://on-gaku.aim-inc.info/index.html
\\

\par
5.Runtastic Pedometer
\par
「機能」
\par
歩数計
\par
音楽アプリの呼び出し
\par
どこでも計測可能:ズボンやジャケットのポケット
\par
スピード測定
\par
歩いた総距離
\par
歩数頻度測定
\par
SNSシェア機能
\par
runtastic.comのフィットネスサイトでのデータ管理
\par
体重,身長のデータ保存によるカロリー計算
\par
ユーザ登録
\par
Google+ ログイン
\par
省エネモード
\par
「類似点」
\par
歩数計測によるテンポ検出
\par
テンポ検出と音楽聴取の同時作業
\par
「差別化ポイント」
\par
別のアプリを呼び出さない
\par
シチュエーション取得
\par
自動選曲
\par
「収益モデル」
\par
「ターゲット層」
\par
iOS,Android
\par
健康管理をサポートして欲しい人
\par
URL
\par
https://www.runtastic.com/ja/apps/pedometer
\\

\par
6.Splyce
\par
「機能」
\par
音楽プレーヤー
\par
プレイリスト
\par
音楽のテンポ検出
\par
音楽のテンポアレンジ
\par
音楽の順番変更
\par
画面の明るさ変化
\par
カメラ用LED点滅
\par
「類似点」
\par
音楽プレーヤー
\par
曲のBPM検出,選曲に利用
\par
「差別化ポイント」
\par
BPMと歩いているテンポを選曲に利用
\par
シチュエーション取得
\par
優先度設定
\par
「収益モデル」
\par
Full feature set(200円)
\par
Fancier Upgrade(100円)
\par
Pro feature set(100円)
\par
「ターゲット層」
\par
iOS
\par
DJプレイをしてみたい人
\par
URL
\par
http://inqbarna.com/mobile-apps/splyce.html

\\
\par
\bunseki{坂本 豊教(未来大)}\par
