\subsection{立石 拓也}
5月
\begin{itemize}
\item プロジェクトに参加
\par長崎大学創生プロジェクトスマホアプリ班として,ミライケータイプロジェクトに参加
\item 合宿に向けたアイディア出し
\par Cool Japanimationの元となるLODを使用した観光支援アプリを発案
\item 長崎大学合宿リーダ就任
\par 第一回合宿の長崎大学合宿リーダとして合宿会議などに参加
\end{itemize}
6月
\begin{itemize}
\item 第一回合同合宿
\par 神奈工で開催された第一回合宿に参加
\item Cool JapanimationのHTML5技術習得リーダ就任
\par 第一回合宿で開発が決定したCool Japanimationの開発言語HTML5班の長崎大学リーダとなる
\item Cool Japanimationの仕様書リーダ就任
\par Cool Japanimationの諸々の仕様書を作成する際の長崎大学仕様書リーダとなる
\item 要求定義書の作成
\par 合同開発を行う未来大と分担して要求定義書を作成
\item 要件定義書の作成
\par 合同開発を行う未来大と分担して要件定義書を作成
\item サービス仕様書の作成
\par 合同開発を行う未来大と分担してサービス仕様書を作成
\end{itemize}
7月
\begin{itemize}
\item 中間報告書の作成
\par 3大学で分担してメンバー全員で中間報告書作成
\item HTML5の技術習得及びCool Japanimationの開発
\par Cool Japanimationを開発する言語HTML5の知識や技術を習得,その後開発に取り掛かる
\item 中間発表のポスタ作成
\par 中間発表用のポスタ作成
\item オープンキャンパス
\par 長崎大学で行われるオープンキャンパスでCool Japanimationの概要を発表
\end{itemize}
8月
\begin{itemize}
\item 長崎大学外でのオープンキャンパス
\par 福岡県の高校で行われたオープンキャンパスでCool Japanimationの概要説明
\item 技術習得の連携
\par Cool Japanimation HTML5班のメンバー内で習得技術の情報共有
\end{itemize}
9月
\begin{itemize}
\item HTML5によるCool Japanimationの開発
\par 本格的にCool Japanimationの開発開始
\item 開発機能の分担
\par 機能ごとに開発するメンバーの振り分けを行い,アカウント機能の開発を開始
\end{itemize}
10月
\begin{itemize}
\item アカウント機能の開発
\par 主にログイン機能と新規アカウント機能を開発
\item サーバに対する技術習得
\par ログイン機能を開発するうえで必要となるサーバの知識及び技術習得
\item 合宿リーダ就任
\par 第二回合宿の長崎大学合宿リーダに就任
\end{itemize}
11月
\begin{itemize}
\item アプリケーションプロトタイプ完成
\par 主要機能のみを含むCool Japanimationプロトタイプ完成
\item 第二回合同合宿
\par 未来大で行われた第二回合宿に長崎大はSkypeを使用しテレビ会議という形で参加
\item アカウント機能プロト型完成
\par ログイン機能及び新規アカウント登録の機能完成
\end{itemize}
12月
\begin{itemize}
\item アカウント機能の微調整
\par UIなどの微調整やマイページ等をどうするかの話し合いを行う
\item モノづくりコンテストへ出展
\par 長崎大学創生プロジェクトスマホアプリ班として創生プロジェクトの締めであるモノづくりコンテストにCool Japanimationを出展
\item プロジェクト最終報告
\par プロジェクトの最終報告書を3大学で分担して作成
\end{itemize}
1月
\begin{itemize}
\item 最終報告書制作
\par これまでの活動を最終報告書にまとめる
\item Cool Japanimation完成
\par Cool Japanimationの正規版完成
\end{itemize}
2月
\begin{itemize}
\item 企業報告会
\par 本プロジェクトの最終目標である企業報告会を行う
\end{itemize}
\bunseki{立石 拓也(長崎大)}
