\subsection{Android班}
前期
\begin{enumerate}
\item 開発担当についての問題
\par
神奈工がAndroidでRhyth/Walkを実装し,長崎大がHTML5でCool Japanimationを実装することになった.また,未来大のiOSはRhyth/Walkを担当し,長崎大と同様未来大のHTML5はCool Japanimationを担当となった.リスクを分散させるためや,未来大のAndroid班がどちらかのアプリに固まった場合,バランスが悪くなるため,未来大のAndroidはRhyth/WalkもCool Japanimationも担当することになった.
\item サーバ班との連携通信方法についての問題
\par
サーバとの通信がどのように行うかを話合い,サーバとの通信方法についてはhttpdを使用する方針になっている.
\item 中間発表会に向けたデモアプリ開発による問題
\par
仕様書や中間発表会のスライド・ポスター作成に力を注ぎ,進行が遅れてしまった.中間発表会では画面遷移ができるデモを用意し,完成度を低めにし中間発表会に間に合わせた.
\end{enumerate}

\par 後期
\begin{enumerate}
\item Androidのバージョンの認識のずれ
\par
メンバ間で話し合い,デモに使用する実機のバージョンも考え統一した.
\item eclipseの不具合
\par
メンバ間で不具合の原因,エラーメッセージなどを共有し,対処した.基本的にはEclipseの再インストールにより解決していた. 
\item 各機能のアルゴリズムの実装の遅れ
\par
歌詞解析,テンポ解析のアルゴリズムの実装がスケジュールよりも遅れてしまうことがあった.メンバ間で相談し合い,優先度をつけ最低限の目標を決め,実装することとなった.
\item 「Cool Japanimation」の同時開発による進捗の遅れ
\par
「Cool Japanimation」との同時開発のため,集中的に開発できず,進捗が遅れてしまうことが多々あった.そのため4人を2人ずつ分け、各担当の機能実装を委託し,集中することによって効率的に開発することになった.
\item サーバ連携実装の遅れ
\par
画面の仕様や機能のアルゴリズムに固執してしまい,サーバ連携が滞ってしまった.合同合宿や最終発表会がありデモも見せるため,サーバ連携を後回しにし機能実装を優先的に進めている.
\bunseki{中司 智朱希(未来大)}
