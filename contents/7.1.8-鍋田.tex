\subsection{鍋田 志木}
\par
活動の概要として,5月には本プロジェクトで開発するアプリケーションのアイディア出しと,
自分が開発を行うプラットフォームを決定した.また,プラットフォーム決定時にサーバ班の
リーダに就任し,6月に行なわれた第一回合同合宿の終了時までリーダを務めた.サーバ班と
してのこれからの動きを認識するためにも,前年度のサーバリーダやアドバイザーである新美
先生へアポイントメントをとり,今後の活動方針についてや技術習得について何から始めれば
良いのかの相談をした.6月では第一回合同合宿が行なわれ,私の班では「しゃべるんです」という
アプリケーション案を提案したが,選ばれることは無く,開発するアプリケーションは
「Cool Japanimation」と「Rhyth/Walk」に決定した.合宿後,サーバ班のメンバー2人はOSSセミナー
に参加し,サーバ構築やApacheやPHPなど,サーバに関連する様々な技術の基本的な内容を
学んだ.アプリケーションに関する仕様書の作成を行うに当たって,私は「Rhyth/Walk」の
詳細仕様書のリーダを担当することになり,6月から7月にかけてスケジュールの作成や目録
の作成を行い,メンバに対して役割分担を行った.7月からは中間発表会の準備が始まり,
「Rhyth/Walk」のスライドと台本,ポスターの作成に力を注いだ.中間発表会の後は中間報告書の
作成を行い,前期の活動についてまとめた.8月から10月にかけては,8月に「Cool Japanimation」
のアプリケーションサーバを構築するのに携わり,このアプリケーションサーバを長崎大の人にも
扱えるよう,アカウントの作成に協力した.その後はアプリケーション開発で用いることを決定
したPHPとMySQLの技術習得をした.11月に行われた第二回合同合宿では,「Rhyth/Walk」の
アプリケーション紹介に必要なスライドの作成を行った.12月の成果発表会でも「Rhyth/Walk」の
アプリケーション紹介に必要なスライドの作成を担当し,プロジェクトメンバで発表練習を何度も
行いその度にスライドをレビューしてスライドを手直しし,より良いスライドを作成することをした.
今回は複数人がこのスライドを利用するため,台本の用意も行った.またアプリケーションを
紹介するに当たって映像による紹介も行おうという考えが出たので,Rhyth/Walkの開発メンバ数人に
協力してもらいスライドに使用する映像の撮影を行い,イメージ映像を作成した.
\par
5月
\begin{itemize}
\item サーバ班に就任
\item サーバ班リーダに就任(6月まで)
\item サーバ班の今後の活動方針を相談
\item サーバ班の活動議事録をとり,Wikiに編集
\item アプリケーションのアイディア発案
\item アイディアのプレゼンテーションスライドの作成
\item プレゼンテーションスライドの台本の作成
\end{itemize}
6月
\begin{itemize}
\item 第一回合同合宿
\item 開発に必要な技術の洗い出しと技術習得
\item OSSセミナーの受講
\item サーバ班の活動議事録をとり,Wikiに編集
\item 「Rhyth/Walk」の要求定義書の作成開始
\item 「Rhyth/Walk」の要件定義書の作成開始
\item 「Rhyth/Walk」のサービス仕様書の作成開始
\item 「Rhyth/Walk」の詳細仕様書リーダに就任
\end{itemize}
7月
\begin{itemize}
\item 「Rhyth/Walk」のアプリケーションサーバ構築
\item 「Rhyth/Walk」の要求定義書のための会議に参加
\item 「Rhyth/Walk」の要件定義書のための会議に参加
\item 中間発表用の「Rhyth/Walk」の台本,ポスタ,スライド作成協力
\item 中間発表
\item 「Rhyth/Walk」の詳細仕様書の作成開始
\item 夏季休暇中の計画停電時に行なうWikiのサーバの停止と起動に関する相談
\item サーバ班で夏季休暇中の活動についての会議
\item サーバ班の活動議事録をとり,Wikiに編集
\item 中間報告書の作成
\end{itemize}
8月
\begin{itemize}
\item 「Cool Japanimation」のアプリケーションサーバ構築
\item オープンキャンパスの用意
\item MySQL・PHPの技術習得
\item アプリケーションサーバのプログラム作成
\end{itemize}
9月
\begin{itemize}
\item アプリケーションとサーバ間の通信方法に関する技術調査
\item MySQL・PHPの技術習得
\item アプリケーションサーバのプログラム作成
\end{itemize}
10月
\begin{itemize}
\item アプリケーションとサーバ間の通信方法に関する技術調査
\item MySQL・PHPの技術習得
\item アプリケーションサーバのプログラム作成
\item サーバ班の活動議事録をとり,Wikiに編集
\item キャンパスベンチャーグランプリにエントリー
\end{itemize}
11月
\begin{itemize}
\item アカデミックリンクに参加
\item 第二回合同合宿へ向けた「Rhyth/Walk」のスライド作成
\item 第二回合同合宿
\item 成果発表会へ向けた「Rhyth/Walk」のスライド・台本作成を担当
\item 成果発表会で使用するスライドのマージ作業担当
\item 「Rhyth/Walk」のロゴコンテストの開催
\item 最終報告書の作成
\end{itemize}
12月
\begin{itemize}
\item 「Rhyth/Walk」のロゴコンテストの開催
\item 「Rhyth/Walk」のロゴ決定
\item 成果発表会へ向けた「Rhyth/Walk」のスライド・台本作成
\item 成果発表会
\item 「Rhyth/Walk」のスライドやポスターに説得力を持たせるための情報収集
\item 最終報告書の作成
\end{itemize}
1月
\begin{itemize}
\item 最終報告書の作成
\item 企業報告会の準備
\item 企業報告会へ向けた「Rhyth/Walk」のスライド・台本作成を担当
\item 秋葉原課外成果発表会の準備
\item サービス仕様書の修正
\item 詳細仕様書の修正
\end{itemize}
2月
\begin{itemize}
\item 秋葉原課外成果発表会の準備予定
\item 企業報告会の準備予定
\item 秋葉原課外成果発表会の開催予定
\item 企業報告会の開催予定
\end{itemize}

\par
開発面に関して,私は5月から6月にかけて,サーバ班のリーダを務めたが,サーバ班に所属した時
,私はサーバについての事前知識はほとんど無く,何からすれば良いのかまるでわからなく,前年度
のサーバリーダやアドバイザーである新美先生に何度も相談に乗ってもらい,今後どのように活動
するかをメンバでよく話し合うようにすることで解決した.また活動しているうちに私はメンバを
引っ張る側には向いていないと思い,このことをメンバに相談して,サーバ班のリーダを変更して
もらい,サーバ班をサポートする立場になった.技術に関しては,OSSセミナーを始めとして技術
習得を行い,徐々に知識を得ていき,操作にもなれていった.わからないことはインターネットを
利用して自力で解決したり,メンバに訊ねることで解決した.サーバとアプリケーション間の通信
方法に関してだが,技術調査を行い参考になりそうなWebページをWikiにあげるなどして情報共有を
はかり,通信方法を確立していこうとしていたがJavaやSwiftに関する知識が無く,技術不足で私
一人では通信テストを行うことができなかった.このことをプロジェクトリーダに相談すると,
Javaとの通信が行えているかを確かめることができるツールを作成してくれ,Androidとの通信を
確認することができ,非常に助かった.
\par
本プロジェクトの第一回合同合宿では開発するアプリケーションが決定したが,アプリケーション案
の候補を一度絞った状態で再度グループ編成を行ったグループでは,アプリケーションの見直しを
グループ全員で行い,夜遅くまでビジネスモデルの考案やスライドの作成に協力し,最終的に開発を
するアプリケーションに選ばれることができた.中間発表会では,アプリケーション開発班一丸と
なってスライドやポスターの作成を行い,より良い発表を行うことができたと思う.その後,第二回
合同合宿ではスライド作成を担当した.また成果発表会でもアプリケーションのスライド作成を担当
し,ここではよりアプリケーションについて伝わるように,伝え方を一から考えて必要な情報は可能
な限り図や表を使うことで一目で伝わるようにした.また,スライドの発表練習と共にスライドの
レビューもし,何度もスライドのブラッシュアップを重ねて当初より遥かに伝わりやすいスライドが
作れたと思う.台本にも言い辛い言葉遣いは控えるようにすることで発表者がより発表しやすい台本
を作れたと思う.これからは企業報告会や秋葉原課外成果発表会が控えており,これに使用する
スライドの作成も担当しているので,企業の方々に伝えるためのスライドを作成することを心がけて
良い形でプロジェクトが終われるように貢献する予定である.

\bunseki{鍋田 志木(未来大)}
