\subsection{安藤 歩美}
\par アプリケーションの開発ではAndroid版「Rhyth/Walk」を担当しており,個人の作業として楽曲の歌詞解析機能を担当した.
\par 神奈工の合宿リーダを担当した.2回に渡る合宿のスケジュールの作成の補佐,資料の作成などを行った.
4月
\begin{itemize}
\item セミナに参加
\end{itemize}
5月
\begin{itemize}
\item Androidアプリケーション製作のための技術習得
\item アプリケーションのアイディア出し
\item アイディア提案シートの作成
\item 第一回合同合宿の資料作成の振り分け
\item 第一回合同合宿の資料作成のサポート
\item 技術習得報告デモの作成
\item 合同合宿リーダ会議
\end{itemize}
6月
\begin{itemize}
\item 第一回合同合宿
\item 合宿グループワークでの進行係
\item 要求定義書の作成
\item 要件定義書の作成
\item 画面遷移図の作成
\item 類似アプリケーション調査
\end{itemize}
7月
\begin{itemize}
\item サービス仕様書の作成
\item 中間報告書の作成
\end{itemize}
8 月
\begin{itemize}
\item 必要な技術を習得
\item アプリケーション開発
\end{itemize}
9 月
\begin{itemize}
\item 必要な技術を習得
\item アプリケーション開発
\end{itemize}
10 月
\begin{itemize}
\item 必要な技術を習得
\item アプリケーション開発
\item 合同合宿リーダ会議
\end{itemize}
11 月
\begin{itemize}
\item アプリケーション完成
\item 合同合宿リーダ会議
\item 第二回合同合宿
\end{itemize}
12 月
\begin{itemize}
\item 学内最終発表の準備
\item 学内最終発表
\item プロジェクト最終報告
\item 最終報告書の作成
\end{itemize}
1 月
\begin{itemize}
\item 最終報告書の作成
\item 最終報告会
\end{itemize}
2 月
\begin{itemize}
\item 企業報告会の準備
\item 企業報告会
\end{itemize}
\par チームでの開発は普段なかなか行わないことであり,良い経験となった.このプロジェクトで学んだことを今後生かしていきたい。反省は,開発の段階でプロジェクトリーダや神奈工メンバに沢山頼ってしまったことである.自分の力だけではないが,担当の機能を実装できた.今後チーム開発をする際には,私もメンバをサポートしていきたい.
\bunseki{安藤 歩美(神奈工)}
