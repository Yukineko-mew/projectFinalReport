\subsection{アプリケーション案の作成}

\par
本項目では,このプロジェクトにおいてどのようにアプリケーションアイディア考案を行ったのかを記述する.
各大学それぞれが,合宿に向けて大学代表となるアプリケーションのアイディアを考案,絞り込みを行った.

\par
未来大は1人につき20個のアプリケーションアイディアを考案し,3つのアイディア班にプロジェクトメンバを
分担して1班の中で6個,4個,2個の順で絞り込み,未来大の中で合計6つの案まで絞り込んだ.
各班はそれらのアイディアについてアイディア提案シートを記入し,最終的に投票で4つのアプリケーションのアイディアを
大学の代表として第一回合同合宿へ持ちよることに決定した.
その過程で,類似アプリケーションの調査を行いながら絞込を行った.また,
それらの大学の代表となったアイディアについては,ビジネスモデルキャンバスを利用してビジネスモデルの考案も行った.

\par
神奈工では, セミナーメンバがアプリケーションのアイディアを1人1つ以上考案しリストアップした.
その後, 神奈工プロジェクトリーダがアイディアをグループ化し, マージが可能な案をマージし,
複合アプリケーションとして新たなアイディアを提案した後, 再度リストアップを行った.
リストアップした各アイディアについて, コンセプトを考えてアイディア提案シートを作成後,洗練を行って選定を行い,
類似アプリケーションの調査を行いつつ, アイディアを1つに絞り,それを大学代表のアイディアとした.
大学代表のアイディアについて,再度コンセプトを練り直し,機能の追加・削除等の洗練を行った.
また, キャッチコピーを考え, ビジネスモデルキャンパスを利用して, ビジネスモデルの考案も行った.

\par 
長崎大は,小林透研究室で取り扱っていた研究テーマの1つを取り上げ,合宿に持ち寄るアイディアの原案とした.
そのアイディアを元に,アイディア提案シートを用いてアイディアの洗練を行い,
ビジネスモデルキャンバスを用いてビジネスモデルの考案を行った.

\par
以上のような過程を経て,各大学が合宿に向けてアプリケーションのアイディアを考案した.
\bunseki{小笠原 佑樹(未来大)}
