\subsection{iOS班}
前期
\begin{enumerate}
\item 開発環境の問題
\par 開発にはiPhone端末が必要であったため,大学から借りる必要があり,開発環境の構築が遅れた.
\item 実機テストの問題
\par iOS Developer Programsの登録に問題があり,実機テストを行うのが遅れた.
\end{enumerate}

後期
\begin{enumerate}
\item Xcodeのバージョンによる問題
\par 開発には9月にXcode6の正式版がリリースされるまでXcode6のベータ版を使用していた.9月に正式版がリリース後Xcodeのアップデートを行う.アップデートを行った結果,正式版とベータ版では仕様が異なっており,ベータ版で開発したソースコードを書き直す必要があった.
\item Swiftの参考資料不足による実装の遅れ
\par Swiftは9月に実装されたばかりであり,参考資料が少なかった.エラーや実装に必要なAPIの記載さえている参考資料が見つからないこともあり実装が遅れてしまった.Objective-C言語のソースコードを書き換えたり別の実装方法を考えるなどして実装を行い,参考資料不足を解決した.
\item 各機能の実装の遅れ
\par 機能の実装がスケジュールよりも遅れてしまうことがあった.メンバ間で相談し合い,優先度をつけ最低限の目標を決め,実装することとなった.
\item サーバ連携実装の遅れ
\par
画面の仕様や機能のアルゴリズムに固執してしまい,サーバ連携が滞ってしまった.合同合宿や最終発表会がありデモも見せるため,サーバ連携を後回しにし機能実装を優先的に進めている.
\end{enumerate}

\bunseki{村上 惇(未来大)}
