\subsection{ビジネスモデル説明}
\par ここでは,Rhyth/Walkのビジネスモデルについて述べる.ビジネスモデルは未来大におけるキャンパスベンチャーグランプリの成果物にてサービス仕様書にあったビジネスモデルを具体化させた.
\par 本アプリケーションの事業拡大には3つのフェーズを考えている.まずはユーザの獲得である.ここではまだ本アプリケーションをインストールしていないユーザの獲得に努める.次に提携企業の獲得である.提携企業はアプリケーション内に表示する広告,またはCMソングを提供する企業を想定している.最後は課金ユーザの獲得である.無課金ではCMソングはサビのワンフレーズのみ再生であるが,曲を購入することでフル再生できるようにする予定である.
\par 収益は3つの方法を考えている.1つ目は音楽によるアフィリエイトである.これは音楽配信サイトと連携し,ユーザがアプリケーション内で音楽を購入した場合に,購入額の数%を配当として受け取れるものである.促進のためにマッチした曲を視聴できるようにする予定である.2つ目は広告収入である.広告にはクリック報酬型を用いる.これをアプリケーションの空いているスペースにバナー広告として設置する.3つ目は地元店舗との提携である.地元店舗はユーザが住んでいる地域の店舗を想定している.ユーザが提携済みの店舗の近くにいると,店舗のCMが流れるものものである.このサービスはアプリケーションの配信開始から2年目で行う予定である.
\par 支出は人件費,初期開発費,宣伝広告費,サーバ維持費,テナント賃貸料の5つを考えている.
\par 収支予測では2年目に年間収支がプラスへ,4年目には総合収支がプラスへ,5年目には総合収支が3000万円となる見込みである.
\bunseki{遠藤 崇(神奈工)}
