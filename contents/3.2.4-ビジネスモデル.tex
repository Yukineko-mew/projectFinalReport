\subsection{ビジネスモデル説明}
\\par
この節では,プロジェクトで企画・開発を行ったアプリケーション「Rhyth/Walk」に対して,考案したビジネスモデルについて説明する.
\par
「Rhyth/Walk」では,「アプリケーション内でのCM放送」と「音楽のアフィリエイト」,「広告による収入」,「音楽レーベルとの提携による音楽販売」といった4つ収益モデルを以下のように作成,考案した.
\par
\begin{itemize}
\itemアプリ内でのCM放送
「Rhyth/Walk」は音楽再生アプリケーションであることから,画面を見ずとも,広告を見せる方法として,地元店舗と提携することによってラジオのようにアプリケーション内でCM放送をすることによって広告としての利益を得ることができる.
\par
\item音楽のアフィリエイト
まずアフィリエイトとは,自分のサイトやブログなどで広告主の商品やサービスを紹介することで,成果があがった場合に報酬を受け取ることができる仕組みのこという.そして音楽のアフィリエイトとは,iTunesなどの音楽配信サイトの音楽を本アプリケーションで紹介することによって,音楽が購入された場合に利益を得ることができる.
\par
\item広告による収入
本アプリケーション内でバナー広告を出し,ユーザにクリックしてもらうことにより,ワンクリック数円の広告収入を得る.
\par
\item音楽レーベルとの提携による音楽販売
これは音楽レーベルと提携することによって,本アプリケーション内でも音楽を販売できるようにすることによって利益を得る.ただし,これは事業が軌道に乗った後のこととなる.なぜなら,アプリケーションで収集したビッグデータを解析したデータを元として,音楽レーベルとの提携を実現させようと思っているためである.つまり,ビッグデータの収集のために,多くの期間が必要だと予想される.そのため,それまでは前述した「音楽のアフィリエイト」を行っていくものとした.
\end{itemize}
\bunseki{三栖 惇(未来大)}
