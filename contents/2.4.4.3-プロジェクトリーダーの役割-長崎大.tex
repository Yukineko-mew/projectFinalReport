\subsubsection{長崎大学}
\par プロジェクトリーダは自分の担当箇所だけでなくメンバ全員の進捗状況を把握しておく必要がある.締切日に間に合わなかった場合はすべてリーダの責任であるので,定期的に様々なタスクの進捗状況を把握してメンバに呼びかけている.またプロジェクト全体のスケジュールと,長崎大のスケジュールを照らし合わせて最も効率の良いスケジュールの組み方を発案するのもプロジェクトリーダの役割である.
\par またプロジェクトリーダーには,実装における技術的な知識も要求される.自分が担当する機能だけではなく,他のすべての機能の技術も学ぶべきである.
\bunseki{※磯野 祐太(長崎大)}
