\subsection{長崎大学}
\par
この節では本プロジェクトにおける長崎大のスケジュールについて述べる.
\par
5月
\begin{itemize}
\item プロジェクト発足及び各リーダ振り分け
\item アプリケーションのアイディア出し
\item 各アイディアに対してのブラッシュアップ
\item 第一回合宿に向けての調整
\item アイディアシートとプレゼンテーション資料の作成
\item ビジネスモデル講演会
\item 学内でのプレゼンテーション発表
\end{itemize}
6月
\begin{itemize}
\item 第一回合同合宿
\item 実装時の技術班の割り振り
\item 要求定義書の作成
\item 要件定義書の作成
\item サービス仕様書の作成
\item アンケート作成
\item 類似アプリケーションの調査
\end{itemize}
7月
\begin{itemize}
\item 中間報告書の担当部分の作成
\item オープンキャンパス
\item 画面遷移図の作成
\item 実装のための技術習得開始
\end{itemize}
8月,9月
\begin{itemize}
\item 実装のための技術習得
\item 要求定義書,要定義書,サービス仕様書,詳細仕様書の見直し
\end{itemize}
10月
\begin{itemize}
\item 実装期間
この実装機関で全体の7割程度は実装できたのではないかと思う.
\end{itemize}
11月
\begin{itemize}
\item 第二回合同合宿
長崎大学は,Skypeによるビデオ通話にて参加した.
\item アプリケーションテスト期間
実際にはテストを行っていない.理由としては実装が間に合っておらず,テストをする段階に至っていないかったからである.
\item 中間報告会
\end{itemize}
12月
\begin{itemize}
\item ものづくりコンテスト
ものづくりコンテストでは惜しくも7位という成績であった.
そもそも出展作品が当作品以外全てハードウェア系だった中で十分検討したと言える.
\end{itemize}
1月
\begin{itemize}
\item 最終報告書の提出
\end{itemize}
2月
\begin{itemize}
\item 企業報告会
\end{itemize}
\bunseki{磯野 祐太(長崎大)}
