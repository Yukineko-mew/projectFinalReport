\subsection{長崎大学}
\par
私たちの前期成果は,アプリケーションの企画・提案を実際に体験し,ビジネスモデルについて学び,各々が責任者となることでメンバ全員が様々な成長をしたことである.具体的な成長としては,技術的な成長と,人間的な成長の2つに分けられる.
 まず技術的な成長としてはWikiの管理,GoogleDriveにおける作業環境共有の手法の会得,HTML5におけるアプリケーション開発のプロセス,サーバの知識取得などがあげられる.長崎大はメンバの半数が情報系の生徒では無いため,ほとんどの技術を一から学ぶことになったが,お互いに協力し合うことで短期間で情報系の技術を身に付けることに成功した.
 次に人間的な成長としては,1つのプロジェクトを共同で,しかも離れた地にいる人同士で進めていくことを通じて様々なことを学んだことがあげられる.共同作業であるため,一人でも期限を守ることができなかったらそれがプロジェクト全体に響いてしまうため,責任感をもって取り組むことができた.また,スケジュール管理を行うことの大切さ,および難しさを学んだ.他大学間など,離れた地にいる人と進捗状況を共有することの難しさを学んだ.始めは会議などで一人の人が受け答えをする形となり進行がスムーズにいかなかったが,一人ひとり積極的に意見を出せるようになったことがよかった.
 反省点としてはWikiを十二分に活用できていなかった所があったので,今後はWikiなどのツールを積極的に活用してさらに進捗状況共有がうまくできるようにしていきたい.

\bunseki{大鶴宗慶(長崎大)}
