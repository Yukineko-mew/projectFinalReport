\subsection{専修大学 渥美幸雄先生の講演}
\par
この節では,専修大学 渥美幸雄先生の講演について述べる.

\begin{description}
 \item[概要]\mbox{}\\
        専修大学の渥美幸雄先生に,既存のビジネスモデルについてや,アイディアを発散・収束させるためのツールについて,今後のプロジェクトの活動に役立ついくつもの貴重な話を学んだ.
 \item[成果]\mbox{}\\ 
        この講演では既存のビジネスモデルのパターンについてや,アイディアを発散・収束させるためのツールについて学んだ.まず,既存のビジネスモデルのパターンを知ることは,ビジネスモデルを考える際にとても大切であり,それぞれのビジネスモデルの特性や,強みや弱みを学ぶことができたことは,ビジネスモデルを考える際に大変役立った.また,アイディアを発散・収束させるためのツールとして,普段から何気なく利用しているブレインストーミングやKJ法についての認識を改めることができたとともに,特性要因図によって原因と結果を系統的かつ階層的に表現する方法や,ブレインライティングというアイディアの出し方を学んだ.これらはプロジェクトでアイディアを発散させたり,整理する時に大変役に立った.また,アンケートについての話では,顧客にアンケートをとっても直接的に今求められていることは出てくるのではなく,その結果の本質を見抜いて対応していくことが大切なのだと学んだ.さらに,一つだけ品質を良くしても他の品質が悪ければそのソフトウェアの品質は良いと評価されず,逆に一つだけ品質が悪ければソフトウェアの信頼性は高くならないので,全体のバランスが大事なのだと学んだ.
\end{description}
\bunseki{澤田 隼(未来大)}
