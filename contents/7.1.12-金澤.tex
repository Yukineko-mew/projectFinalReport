\par
活動内容と予定
\par
5月はプロジェクトで開発するアプリケーション案を考案し,ブラッシュアップでアプリケーション案を絞込み,
合宿に向けてスライド作成を行った.
6月には第一回合宿を行い,アプリケーションが決定した.
開発のプラットフォームはHTML5になったのと同時にHTML5班のリーダに就任した.
7月は学内で中間発表会が行われた.
中間報告書を書いて提出するため,グループ報告書のリーダに就任した.
8,9月はCool Japanimationのメンバのみで会議を開き,開発する機能の分担をした.
10月はキャンパスベンチャーグランプリに参加するためコンペティション書類作成を行った.
11月はアカデミックリンクに参加するため,デモの開発を行った.
12月は成果発表会があったので,発表練習とレビューを重ねた.
最終報告書の提出に向けて,再びグループ報告書リーダに就任した.
1月で作成したアプリケーションのテストを行う.
また,2月には本プロジェクトの最終ゴールである,企業報告会がある.これからは,企業報告会に向け,スライド
や,ポスターなどの発表資料の改良や,アプリケーション自体の改良を行っていく予定である.
\subsection{金澤 しほり}
5月
\begin{itemize}
\item アプリケーション案の発案
\item 会議やブラッシュアップでのアプリケーション案の絞り込み
\item グループで会議,スライド作成
\item Skype会議
\item  他大学や未来大内で話し合いを行う必要があるため
\end{itemize}
6月
\begin{itemize}
\item 第一回合同合宿を実施
\item 用意した案の発表で質疑応答
\item グループでのCool Japanimationの検討
\item 合宿で決定した案を元に要求定義書,要件定義書,サービス仕様書の作成
\item HTML5技術習得リーダ就任
\item Skype会議
\item  他大学や未来大内で話し合いを行う必要があるため
\end{itemize}
7月
\begin{itemize}
\item HTML5技術習得
\item Skype会議
\item  他大学や未来大内で話し合いを行う必要があるため
\item 中間発表会
\item 学内でプロジェクトの中間発表会が
\item 中間報告書について
\item   中間報告書のリーダー担当
\item   中間報告書の担当割り振り
\item  スケジュール作成
\item   中間報告書の作成
\item サービス仕様書
\item 詳細仕様書の作成
\end{itemize}
8月
\begin{itemize}
\item Cool Japanimation-LINE会議
\item   他大学や未来大内で話し合いを行う必要があるため
\item 機能担当分担
\item  会議の際に,各機能の開発の担当決め
\item レビュー機能開発
\item   レビュー機能の担当就任
\end{itemize}
9月
\begin{itemize}
\item レビュー機能開発
\item  レビュー機能の担当就任
\end{itemize}
10月
\begin{itemize}
\item Cool Japanimation-LINE会議
\item   他大学や未来大内で話し合いを行う必要があるため
\item Skype 会議
\item 他大学や未来大内で話し合いを行う必要があるため
\item レビュー機能開発
\item   レビュー機能の担当就任
\item 機能のマージ作業
\item   各機能が単体で出来上がったので,その機能をマージして一つのアプリケーションで使用できるようにすること
\item キャンパスベンチャーグランプリ
\item   キャンバスベンチャーグランプリにエントリーするため,
\item   アプリケーションの説明をドキュメント化し,ビジネスモデルの考案
\end{itemize}
11月
\begin{itemize}
\item Cool Japanimation-LINE会議
\item  他大学や未来大内で話し合いを行う必要があるため
\item Skype会議
\item   他大学や未来大内で話し合いを行う必要があるため
\item 機能のマージ作業
\item   各機能が単体で出来上がったので,その機能をマージして一つのアプリケーションで使用できるようにすること
\item アカデミックリンク
\item   アカデミックリンクに参加し,アプリケーションのデモが行えるようにデモ用に作成
\item 第二回合同合宿
\item   合宿用にCool Japanimationの進捗スライドの作成
\end{itemize}

12月
\begin{itemize}
\item Skype 会議
\item   他大学や未来大内で話し合いを行う必要があるため
\item ツアー参加申請機能提案
\item    ツアー参加申請を新たに付け足すことにしたため,画面遷移図などを考案
\item ツアー参加申請機能開発
\item    ツアー参加申請機能の画面遷移図などを元に開発
\item プロジェクト成果発表会
\item    学内でプロジェクト成果発表会
\item    プロジェクト成果発表用のデモ作成
\item 最終報告書
\item   最終報告書リーダー就任
\item   最終報告書のスケジュール作成
\item  三大学分の担当割り振り
\item   最終報告書作成
\end{itemize}

1月
\begin{itemize}
\item Skype 会議
\item   他大学や未来大内で話し合いを行う必要があるため
\item Cool Japanimation-LINE会議
\item   他大学や未来大内で話し合いを行う必要があるため
\item ツアー参加申請機能開発
\item   ツアー参加申請機能の画面遷移図なを元に開発作業
\item 機能のマージ作業
\item   各機能が単体で出来上がったので,その機能をマージして一つのアプリケーションで使用できるようにすること
\item 最終報告書の作成
\item 企業報告会の準備
\end{itemize}

2月
\begin{itemize}
\item ツアー参加申請機能開発
\item 機能のマージ作業
\item Cool Japanimation-LINE会議
\item   他大学や未来大内で話し合いを行う必要があるため
\item Skype 会議
\item   他大学や未来大内で話し合いを行う必要があるため
\item 企業報告会の準備
\item 企業報告会
\end{itemize}

\par 
活動内容
\par
私は本プロジェクトのアプリケーション開発で,Cool Japanimationを担当することになった.
HTML5班に所属し,HTML5の開発班のリーダーを務めた.
開発環境はMonacaをインストールして開発を行った.
言語はHTMLとJavascriptであった.HTMLは以前にも授業で触れたことがあったが,
Javascriptは初めてであったので,自分の担当した機能の設計詳細を考え,
Web上のサンプルプログラムと解説が載っているページを見つけて勉強するようにした.
夏休みのときに割り振られた自分の担当した機能は無事完成を迎えることができた.
後期で機能のマージ作業に入ったが,原因がわからず一部機能が動かなかった.
途中でどうしてもやり方が分からない処理もあったが,
その問題をミライケータイプロジェクトのメンバに相談しながら一緒に解決することができた.
日々の活動の中では,メンバ間で交代し議事録を取った.
わかりやすい議事録を取るのを心がけ,細かいと
ころまで記述するようにした.
また,中間と最終のグループ報告書のリーダーも担当し,
決められた条件を達するように他大学の担当割り振りをバランスよくなるように行い,
期日までに良いものを仕上げるためにスケジュールの管理も心がけた.
\bunseki{金澤 しほり(未来大)}
