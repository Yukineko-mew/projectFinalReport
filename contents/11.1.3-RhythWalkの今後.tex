\subsection{Rhyth/Walkの今後}
\par
今後,アプリケーション「Rhyth/Walk」では以下のことを行う.

\begin{description}
 \item[各仕様書の修正]\mbox{}\\ 
	    今まで作成してきた各 4 つの仕様書には不備や変更点などがある.そのため,グループで分担し,効率的に仕様書の修正を行う.
	    具体的な変更点は,アプリ名や機能の詳細などである.
	    
 \item[デモシナリオの見直し]\mbox{}\\
	    企業発表会に向けてアプリケーションのアピールポイントを明確にするという意味でもデモシナリオの見直しを行う.
	    そして,そのデモシナリオを参考にしながら企業発表会のスライドの作成,秋葉原課外発表会のポスターを作成する.

 \item[機能の未完成部分の実装]\mbox{}\\
	    AndroidとiOSの2つのプラットフォームで開発を行っているが,未完成の機能がある.
	    今後この未完成の機能を実装し,AndroidとiOSの両プラットフォームで「Rhyth/Walk」のアプリケーションの完成を目指す.

\item[楽曲解析の精度の向上]\mbox{}\\
	    現在「Rhyth/Walk」では,音楽のタイトルと歌詞から楽曲解析を行っている.
	    しかし,この解析方法では歌詞のある曲のみに限定され,歌詞に書かれている情報からしか音楽のシチュエーションを解析することができない.
	    その解決策として,音楽音響信号からスケールやコード進行の取得を試みているが,ピアノソロ曲であればスケールを判定できるというレベルにとどまっている.
	    今後はさらに改良し,ピアノソロ曲以外でもスケールやコード進行を取得できるようにし,音楽のシチュエーションの解析の精度の向上を試みる.
	    その他のアプローチとしては,LODを用いて歌詞からは得られない音楽の背景を取得し,精度の向上を試みる.
   
\item[各種印象調査]\mbox{}\\
	    現在「Rhyth/Walk」では,プロジェクトメンバの主観に基づいて現在の季節や時間帯を分別している.
	    また,歌詞解析で使用されるシチュエーションに関する単語の決定や,その重み付けも主観に基づいて決められている.
	    そこで,ターゲットユーザに単語の印象などの調査を行い,その結果を「Rhyth/Walk」に反映させることで,ターゲットユーザから,より需要の高いアプリを作成する.
	    さらに,スケールやコード進行がシチュエーションへ与える印象についても調査し,よりシチュエーションに一致した音楽の再生を試みる.
   
\item[サーバ連携]\mbox{}\\
	    「Rhyth/Walk」はサーバ連携が不十分である.そのため,サーバ連携を行い,アプリケーションの完成を目指す.

\item
[Rhyth/Walkアプリケーションのテスト]\mbox{}\\
	    「Rhyth/Walk」では,未完成の機能の実装後,テスト項目の作成とテストを実施する.
	    テストを実施する目的はアプリケーションのバグを発見するためである.

\item[ソースコードの修正]\mbox{}\\
	    テストの実施により,発見されたバグや,不足している箇所の修正を行う.
	    また,ソースコードの最適化も同時に行い,メモリ効率などを考える.
	    これらの行程により,完成度の高いアプリケーションを作成する.

\item[秋葉原課外活動の準備]\mbox{}\\
	    本発表はポスターセッション形式で発表するため,発表に使用するポスターの作成を行う.
	    「Rhyth/Walk」のポスターとしては,アプリケーションの概要や機能説明,ビジネスモデルを書いたポスターを用意する予定である.

\item[企業報告会準備]\mbox{}\\
	    サポート企業に出向き企業報告会を行うので,そのための発表資料や発表手順などを考案し,作成する.
	    サポート企業に今までのプロジェクトの成果を発表できる貴重な機会であるので,どのような説明を行えばよいか考え,準備を行う.
	    準備の主な内容は,スライドの作成とアプリケーションの開発である.

\item[成果物の整理]\mbox{}\\
	    プロジェクトでは,サポート企業に成果物のDVDを作成する.
	    「Rhyth/Walk」は,ソースコードの行数や,作業時間などを考慮し,工数を計算したり,作成した「Rhyth/Walk」に関するすべての成果物を提出する準備を行う.

\end{description}

\bunseki{澤田 隼(未来大)}
