\subsection{紺井 和人}
\par
活動の概要として,5月には,今回開発するアプリケーションのアイディア出しと,自分が開発するためのプラットフォームを決めた.そして,6月は第一回合宿があり,そこで開発するアプリケーション2つを決定し,自分は,「Cool Japanimation」を開発することになった.その後,各仕様書のリーダ決めを行い,自分は詳細仕様書のリーダに就任した.そして4つの仕様書のうち,要求定義書と要件定義書の作成を行った.7月は,中間発表向けての準備を行い,自分は「Cool Japanimation」のサブポスタの作成を担当した.そして,中間発表で,スライド発表とポスタセッションを行った.8月,9月は,夏休みの間はアプリケーションの開発に向けての技術習得,自分の場合は,Android開発の基本的な技術習得を行った.9月後半からは,キャンパスベンチャーグランプリのリーダに就任した.そしてキャンパスベンチャーグランプリに向けての書類作成を行い,ビジネスモデルを考案した.その中でも自分は収益予測を担当した.10月は,キャンパスベンチャーグランプリの評価,修正を繰り返し,提出した.11月はアカデミックリンクがあり,自分はアカデミックリンクの代表に就任し,アカデミックリンクに向けての準備特に,デモのためにアプリケーションの開発を行った.また,第二回の合宿があり,自分は,「Cool Japanimation」のアプリケーションを紹介するスライドを作成した.開発に関しては,合宿で使用するデモのために,自分の担当である,アカウント機能の画面遷移を実装した.12月は最終発表があり,それに向けての準備として,自分は中間発表に引き続き「Cool Japanimation」のサブポスタを担当した.開発に関しては,デモに使うアプリケーションに,ツアーの申請,許可機能の画面遷移の実装をアカウント機能に次いで担当し,実装した.また,最終報告書の作成にとりかかった.1月は、2月に行われる秋葉原での課外成果発表のポスタセッションと企業報告会に向けての準備を行い,自分は2月に秋葉原での課外成果発表会に参加する予定である.
\par
5月
\begin{itemize}
\item 第一回合宿に向けたアイディア出し
\item Android技術習得に参加
\item 第一回合同合宿デモ作成(Android)
\end{itemize}
6月
\begin{itemize}
\item 第一回合宿に向けたアイディアのブラッシュアップ
\item 第一回合宿で発表するアイディアの発表スライドの作成
\item 第一回合同合宿
\item Android班「Cool Japanimation」リーダに就任
\item 詳細仕様書リーダに就任
\item 「Cool Japanimation」要求定義書の作成
\item 「Cool Japanimation」要件定義書の作成
\end{itemize}
7月
\begin{itemize}
\item 要求定義書のための会議に参加
\item 要件定義書のための会議に参加
\item サービス仕様書のための会議に参加
\item Tex講座に参加
\item 中間発表用のサブポスターを作成
\item 中間発表用のスライド作成協力
\item 中間発表用のメインポスター活動の作成
\item 中間発表用デモ作成(Android)
\item 中間発表
\end{itemize}
8月
\begin{itemize}
\item アプリケーション開発
\item 必要な技術の習得
\end{itemize}
9月
\begin{itemize}
\item アプリケーションの開発
\item 必要な技術の習得
\item キャンパスベンチャーグランプリ担当者就任
\item キャンパスベンチャーグランプリ,収支予測作成
\end{itemize}
10月
\begin{itemize}
\item アプリケーション開発
\item デバッグ
\item 最終発表の準備
\item キャンパスベンチャーグランプリ,評価修正
\end{itemize}
11月
\begin{itemize}
\item アカデミックリンク代表者就任
\item アカデミックリンクの準備
\item アカデミックリンク
\item アプリケーションの開発
\item デバッグ
\item 第二回合宿に向けてのアプリ紹介スライドの作成
\item 第二回合宿
\end{itemize}
12月
\begin{itemize}
\item プロジェクト最終発表会準備
\item プロジェクト最終発表会用のサブポスタ作成
\item プロジェクト最終発表会
\item 最終報告書作成
\end{itemize}
1月
\begin{itemize}
\item 最終報告書作成
\item 秋葉原での課外成果発表会の準備
\item 企業報告会の準備
\end{itemize}
2月
\begin{itemize}
\item 秋葉原での課外成果発表会の準備
\item 秋葉原での課外成果発表会
\item 企業報告会の準備
\end{itemize}
\par
今回Android班として,Android開発をして,苦労した点はAndroidはバージョンがよく上がり,そのたびに何かしらのソースコードの仕様変更があるので,バージョンによって正常に動かせてコードが,動かせなくなり,そのバージョンにあわせたコードの書き方をしなければならず,大変だった.しかし,エラーが起こるたびに,エラーメッセージを読み,エラー箇所を理解し,どうすれば解決できるかを調べ,解決していくという開発者にとっては当たり前の作業だが,この先SEになる上でとても大切なスキルを身につけることができた.
\par
自分はキャンパスベンチャーグランプリのリーダとして,自分の担当である収支予測だけでなく,このアプリケーションをどうやって売るかというビジネスモデル全体を学習した.最初は知識が全くなかったので,過去のミライケータイプロジェクトのOB,OGのキャンパスベンチャーグランプリの書類を参考にしたり,調べたりして,評価を何度かもらい修正していくうちに改善されていった.改善していくにあたって,自分の担当するアプリケーションをどうやって売るか,どうすればアプリケーションのユーザ数を増やすか,起業した場合どれくらいのお金が動くかを考える難しさを改めて知った.

\bunseki{紺井 和人(未来大)}
