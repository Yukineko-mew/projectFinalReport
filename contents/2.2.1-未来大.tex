\subsection{公立はこだて未来大学}
\par この節では,プロジェクト学習を効率よく遂行するために設定した課題について述べる.
\par ●前期
\par 前期はアプリケーションの企画,設計,ビジネスモデルの考案のため以下の課題を設定した.
\begin{enumerate}
\item 各プラットフォームの特徴の調査
\par アプリケーション開発を行うプラットフォームを調べ,そのプラットフォームの特徴を調査すること.
\item アプリケーション開発技術の習得 
\par 開発するアプリケーションに必要な技術を習得すること.
\item 開発するアプリケーションのアイディア企画
\par 各学校のグル―プでいくつかアプリケーションのアイディアを持ち寄り,その後,第一回合同合宿にて絞り込み,3大学混合グループでアイディアを練り直し,投票で開発するアプリケーションを決定すること.
\item ビジネスモデル作成技術の習得
\par 開発するアプリケーションのビジネスモデルを作成するための技術知識の習得を行うこと.
\item 類似サービスの調査
\par 第一回合同合宿で決定したアプリケーションが提供するサービスに対して,同じようなジャンルのサービスはどのようなものかを調査すること.\item 開発アプリケーション名の決定
\par 第一回合同合宿で決定したアプリケーションに,アプリケーションの内容を踏まえた上で,考察し名前を決めること.
\item 要求定義書の作成
\par 開発アプリケーションがどのような仕様を設けるのかを記したものを,3大学で項目ごとに分担して合同で作成すること.
\item 要件定義書の作成
\par 開発アプリケーションのソフトウェア要件をまとめたものを,3大学で項目ごとに分担して合同で作成すること.
\item サービス仕様書の作成
\par 開発アプリケーションが,ユーザの視点でどのようにサービスを提供するのかを記したものを,3大学で項目ごとに分担して合同で作成すること.
\item 詳細仕様書の作成
\par アプリケーションに実装されている機能を詳細にまとめたものを,3大学で項目ごとに分担して合同で作成すること.
\item アプリケーションの機能選定
\par 合同会議などでアイディアとして出てきた機能の中から,実装する機能の絞り込みを行うこと.
\item  前期提出物の作成
\par 前期期間内の活動内容を,各校で項目を分担して1 つの報告書としてまとめて作成すること.
\end{enumerate}

\par ●後期
\par 後期は前期で決定した事項をもとにアプリケーション機能実装,ビジネスモデル構築を行うために以下の課題を設定した.
\begin{enumerate}
\item サーバ処理の実装
\par 開発するアプリケーションの機能に必要なサーバ側の処理を実装すること.
\item アプリケーションの機能の実装
\par 選定した機能について,優先度が高い順に実装すること.
\item ビジネスモデル作成技術の習得
\par 開発するアプリケーションを用いたビジネスモデルを作成するための技術習得を行うこと.
\item ビジネスモデルの構築
\par ビジネスモデルの定義,アプリケーションの機能,アンケート結果などを参考にビジネスモデルを構築すること.
\item キャッチコピー,デモシナリオの作成
\par アプリケーションの利用方法やそのアプリケーションイメージを端的に伝えるよう作成すること.
\item 最終報告書の作成
\par 後期期間内の活動内容を,3大学で項目を分担して一つの報告書としてまとめて作成すること.
\end{enumerate}
\bunseki{村上 惇(未来大)}
