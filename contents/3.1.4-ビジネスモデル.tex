\subsection{ビジネスモデル説明}
\par
 ここでは,ビジネスモデルについて述べる.本アプリケーションのビジネスモデルは,大きく分けて3つの収益モデルによって構成されている.それを以下に示す.
\begin{enumerate}
\item 本アプリケーション内のバナー広告によるクリック型広告収入
\par 
本アプリケーション内でバナー広告をだし,ユーザにクリックしてもらうことにより,ワンクリック数円の広告収入を得る.また,バナー広告をアニメ関連のものにすることにより,ユーザにさらなる本アプリケーションの使用意欲をかきたてさせるとともに,クリック数を増やす.
\item 有料コンテンツによる追加課金 
\par
 本アプリケーションのアニメ検索機能では,課金することで,検索したアニメをお気に入りとして登録できる,お気に入り機能が使えるようになったり,チャット機能で,本アプリケーション限定で使えるスタンプなどが使えたりできるようになる.
\item 提携店舗とのクーポンによる広告収入
\par
 訪日を終えた後に,航空券などの日本に行ったことを証明できるものをアプリケーションに提示すると,提携店舗で使用できるクーポン券が発行される.そのクーポンは次回訪日する際に,ユーザは使用できる.また,提携店舗は,アニメイトなどのアニメ関連店舗や,イベントや聖地付近の宿泊施設である.また,クーポンとは,アニメ関連店舗であれば,限定グッツがもらえるものであったり,宿泊施設などでは,割引券などのことである.このクーポンを広告料として,提携店舗から収入を得る.
\par
以上が,12月段階でのビジネスモデル案の説明である.
\end{enumerate}
\bunseki{紺井 和人(未来大)}
