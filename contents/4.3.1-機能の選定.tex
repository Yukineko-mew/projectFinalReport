\subsection{機能の選定}
\par
機能についての基本的なものに関しては,未来大と長崎大がそれぞれ持ち寄った案が元になっている. 
そして第一回合宿で未来大,神奈工,長崎大の三校の代表メンバで話し合いをはかり基礎的な
ものに関しては機能が決定した.
機能の基本的なものに関しては, 未来大と長崎大がそれぞれ持ち寄った案が元になっている.
そして第一回合宿で未来大, 神奈工, 長崎大の3大校の代表メンバで話し合いを図り,基礎的な
ものに関しては機能が決定した. 途中の段階ではアプリケーションの目的や機能の再考案が多く,
上手な融合を図ることが困難であった.
合宿で機能選定したものを元に,各大学での話し合いを行ったり,Skypeチャット,あるいはLINEで会議の場を設けて話し合った.
前期は要求定義書, 要件定義書を作成し,案を詰め, サービス仕様書にて一度は一定の機能を絞り込むことで落ち着いた.
しかし,夏休みで会議を開き,機能を分担して開発を進めていくに連れ,仕様書の段階では知識がなく書けなかった部分も出てきた.
各機能の画面遷移図ばかり考えていたため,トップページやマイページの画面が抜けていたのである.
また,後期後半になってから,ツアー参加申請の画面遷移や,参加申請までの流れが考えれていなかった.
そこで,毎週水曜日に行われる合同会議の残りの時間でCool Japanimationの担当メンバのみで話し合いを行った.
その際に曖昧になっていた機能を説明し,新たに開発担当を決め内容を詰めていくことにした.
現在もまた,ナビ機能を必要と思い開発をしていたが,発表がある場でデモを行うにあたり,ナビ機能は必要ではないと話し合いで判断した.
そのため,ナビ機能の開発は進んではいたものの,Cool Japanimationの機能としては除外することにした.
しかし,また,デモシナリオの内容の見直しを何度もしていくうちに,やはりナビ機能は必要なのではないかと意見が出ているので,これからもまた話し合いを行い決める予定である.
\bunseki{金澤 しほり(未来大)}
