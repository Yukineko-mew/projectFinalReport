\subsection{HTML5班}
\par この節では,本プロジェクトのアプリケーションである「Cool Japanimation」を開発するために行ったプロセスについて述べる.

\par 前期のプロセスを述べる

\begin{itemize}
\item HTML5アプリケーション開発のための環境構築
\item アプリケーション開発のための環境構築,技術習得

\par 技術習得の成果

\par 以下の3つの機能を実装したデモアプリケーションを開発した.(未来大)
\begin{itemize}
\item 写真機能
\item GPS 機能
\item 加速度機能
\end{itemize}

\par ウェアラブルデバイスからネット上に情報を上げることを利用したアプリケーションを開発した.(長崎大)
\end{itemize}

\\
\par 後期のプロセスを述べる

\begin{itemize}
\item 開発スケジュールの決定
\item 決定した画面遷移をもとに以下の機能を開発
\begin{itemize}
\item アカウント機能
\item ツアー作成・検索機能
\item アニメ検索機能
\item チャット機能
\item ナビ機能
\end{itemize}
\item サーバとの連携に向けた開発
\item 未実装機能の優先度を決定
\item 優先度を元に実装スケジュールを決定
\item 未実装機能の開発
\end{itemize}

\bunseki{藤原 由美恵(未来大)}
