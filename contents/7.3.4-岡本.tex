\subsection{岡本 優介}
5月
\begin{itemize}
\item プロジェクトに参加
\item 合宿に向けたアイディア出し
\par アプリの概要やコンセプトについて考えた.
\item 外国人に対してのアンケート調査
\par 外国に住んでいる友達とその知り合いにアプリのアンケート調査を行った.
\end{itemize}
6月
\begin{itemize}
\item 「Cool Japanimation」のHTML班に参加
\item 「Cool Japanimation」の要求定義を議論
\par 要求定義の内容を話し合った.マッチング成功時の連絡先通知機能を担当した.
\item 要求定義書の作成
\par 決められた担当箇所の定義書を作成した.
\item 「Cool Japanimation」の要件定義を議論
\par 要件定義の内容を話し合った.マッチング成功時の連絡先通知機能を担当した.
\item 要件定義書の作成
\par 決められた担当箇所の定義書を作成した.
\end{itemize}
7月
\begin{itemize}
\item サービス仕様書の作成
\par お気に入り機能を担当した.画面遷移図を作成した.
\item 中間報告書の作成
\par 類似アプリケーションの調査を行い,担当箇所の報告書を作成した.
\item 「Cool Japanimation」のアカウント機能を担当
\par monacaをダウンロードし開発環境を整え,Javascriptについて学習を始めた.
\end{itemize}
8月
\begin{itemize}
\item HTML5による「Cool Japanimaition」の開発
\par アカウント機能の開発を進めた.
\end{itemize}
9月
\begin{itemize}
\item HTML5による「Cool Japanimaition」の開発
\par アカウント機能の開発を進めた.
\end{itemize}
10月
\begin{itemize}
\item HTML5による「Cool Japanimaition」の開発
\par アカウント機能の開発を進めた.
\end{itemize}
11月
\begin{itemize}
\item 創成プロジェクトの中間発表
\item	第二回合宿の参加(Skypeにて)
\par Skypeで第二回合宿に参加した.
\item 創成プロジェクトの発表の準備
\par スライド,ポスター,デモ,アプリケーションの紹介の練習を行った.
\end{itemize}
12月
\begin{itemize}
\item 創成プロジェクト発表
\item 「Cool Japanimation」の最終報告書の議論
\par 「Cool Japanimation」の類似アプリケーション調査を担当することになった.
\item 最終報告書の作成
\par 担当箇所を作成した.
\item 最終報告書の第一次案の修正・添削
\par 学生内でレビューを行い,修正を加える予定.
\end{itemize}
1月
\begin{itemize}
\item 最終報告書第一次案の修正・添削
\par 先生方から意見をもらい,修正を加え第二次案を作成する予定.
\end{itemize}
2月
\begin{itemize}
\item 企業報告会
\end{itemize}
\par
活動内容
\par
私はHTML5班の一員としてプロジェクトに参加した.アカウント機能を担当し,HTML5について一から学習を始めた.しかし情報系の人との理解
の差が大きく,多くの機能開発を任せることになってしまった.外国人に対するアンケートを行った所,やはり日本人が何気なく思っているこ
とも外国人から見ると魅力的に感じることがあるなど,いろんな視点から物事を捉えることの大切さに気づいた.始めはしっかりアプリの構想
を考えていたつもりだったが,開発を進めていくうちに変更したほうが良い所や,実装が難しい所などがでできて,見積もりをすることや開発の
難しさを学んだ.第二回合宿会議ではSkypeによる参加で,離れた場所から連絡を取り合うことの難しさを知った.機械系なので今回のプロジェ
クトにおける技術的な所は今後活かせる機会があるかはわからないが,スケジュールの大切さや進捗状況をこまめに確認することでスケジュー
ルの遅れなどに対応し,柔軟に対応していくことなどをこれからの研究などに活かしていきたい.また,企業報告会に向けて担当している機能の
改良を行う予定である.
 
\bunseki{岡本 優介(長崎大)}
